% Options for packages loaded elsewhere
\PassOptionsToPackage{unicode}{hyperref}
\PassOptionsToPackage{hyphens}{url}
\PassOptionsToPackage{dvipsnames,svgnames,x11names}{xcolor}
%
\documentclass[
  letterpaper,
  DIV=11,
  numbers=noendperiod]{scrreprt}

\usepackage{amsmath,amssymb}
\usepackage{iftex}
\ifPDFTeX
  \usepackage[T1]{fontenc}
  \usepackage[utf8]{inputenc}
  \usepackage{textcomp} % provide euro and other symbols
\else % if luatex or xetex
  \usepackage{unicode-math}
  \defaultfontfeatures{Scale=MatchLowercase}
  \defaultfontfeatures[\rmfamily]{Ligatures=TeX,Scale=1}
\fi
\usepackage{lmodern}
\ifPDFTeX\else  
    % xetex/luatex font selection
\fi
% Use upquote if available, for straight quotes in verbatim environments
\IfFileExists{upquote.sty}{\usepackage{upquote}}{}
\IfFileExists{microtype.sty}{% use microtype if available
  \usepackage[]{microtype}
  \UseMicrotypeSet[protrusion]{basicmath} % disable protrusion for tt fonts
}{}
\makeatletter
\@ifundefined{KOMAClassName}{% if non-KOMA class
  \IfFileExists{parskip.sty}{%
    \usepackage{parskip}
  }{% else
    \setlength{\parindent}{0pt}
    \setlength{\parskip}{6pt plus 2pt minus 1pt}}
}{% if KOMA class
  \KOMAoptions{parskip=half}}
\makeatother
\usepackage{xcolor}
\setlength{\emergencystretch}{3em} % prevent overfull lines
\setcounter{secnumdepth}{5}
% Make \paragraph and \subparagraph free-standing
\ifx\paragraph\undefined\else
  \let\oldparagraph\paragraph
  \renewcommand{\paragraph}[1]{\oldparagraph{#1}\mbox{}}
\fi
\ifx\subparagraph\undefined\else
  \let\oldsubparagraph\subparagraph
  \renewcommand{\subparagraph}[1]{\oldsubparagraph{#1}\mbox{}}
\fi


\providecommand{\tightlist}{%
  \setlength{\itemsep}{0pt}\setlength{\parskip}{0pt}}\usepackage{longtable,booktabs,array}
\usepackage{calc} % for calculating minipage widths
% Correct order of tables after \paragraph or \subparagraph
\usepackage{etoolbox}
\makeatletter
\patchcmd\longtable{\par}{\if@noskipsec\mbox{}\fi\par}{}{}
\makeatother
% Allow footnotes in longtable head/foot
\IfFileExists{footnotehyper.sty}{\usepackage{footnotehyper}}{\usepackage{footnote}}
\makesavenoteenv{longtable}
\usepackage{graphicx}
\makeatletter
\def\maxwidth{\ifdim\Gin@nat@width>\linewidth\linewidth\else\Gin@nat@width\fi}
\def\maxheight{\ifdim\Gin@nat@height>\textheight\textheight\else\Gin@nat@height\fi}
\makeatother
% Scale images if necessary, so that they will not overflow the page
% margins by default, and it is still possible to overwrite the defaults
% using explicit options in \includegraphics[width, height, ...]{}
\setkeys{Gin}{width=\maxwidth,height=\maxheight,keepaspectratio}
% Set default figure placement to htbp
\makeatletter
\def\fps@figure{htbp}
\makeatother
% definitions for citeproc citations
\NewDocumentCommand\citeproctext{}{}
\NewDocumentCommand\citeproc{mm}{%
  \begingroup\def\citeproctext{#2}\cite{#1}\endgroup}
\makeatletter
 % allow citations to break across lines
 \let\@cite@ofmt\@firstofone
 % avoid brackets around text for \cite:
 \def\@biblabel#1{}
 \def\@cite#1#2{{#1\if@tempswa , #2\fi}}
\makeatother
\newlength{\cslhangindent}
\setlength{\cslhangindent}{1.5em}
\newlength{\csllabelwidth}
\setlength{\csllabelwidth}{3em}
\newenvironment{CSLReferences}[2] % #1 hanging-indent, #2 entry-spacing
 {\begin{list}{}{%
  \setlength{\itemindent}{0pt}
  \setlength{\leftmargin}{0pt}
  \setlength{\parsep}{0pt}
  % turn on hanging indent if param 1 is 1
  \ifodd #1
   \setlength{\leftmargin}{\cslhangindent}
   \setlength{\itemindent}{-1\cslhangindent}
  \fi
  % set entry spacing
  \setlength{\itemsep}{#2\baselineskip}}}
 {\end{list}}
\usepackage{calc}
\newcommand{\CSLBlock}[1]{\hfill\break\parbox[t]{\linewidth}{\strut\ignorespaces#1\strut}}
\newcommand{\CSLLeftMargin}[1]{\parbox[t]{\csllabelwidth}{\strut#1\strut}}
\newcommand{\CSLRightInline}[1]{\parbox[t]{\linewidth - \csllabelwidth}{\strut#1\strut}}
\newcommand{\CSLIndent}[1]{\hspace{\cslhangindent}#1}

\KOMAoption{captions}{tableheading}
\makeatletter
\@ifpackageloaded{bookmark}{}{\usepackage{bookmark}}
\makeatother
\makeatletter
\@ifpackageloaded{caption}{}{\usepackage{caption}}
\AtBeginDocument{%
\ifdefined\contentsname
  \renewcommand*\contentsname{Table of contents}
\else
  \newcommand\contentsname{Table of contents}
\fi
\ifdefined\listfigurename
  \renewcommand*\listfigurename{List of Figures}
\else
  \newcommand\listfigurename{List of Figures}
\fi
\ifdefined\listtablename
  \renewcommand*\listtablename{List of Tables}
\else
  \newcommand\listtablename{List of Tables}
\fi
\ifdefined\figurename
  \renewcommand*\figurename{Figure}
\else
  \newcommand\figurename{Figure}
\fi
\ifdefined\tablename
  \renewcommand*\tablename{Table}
\else
  \newcommand\tablename{Table}
\fi
}
\@ifpackageloaded{float}{}{\usepackage{float}}
\floatstyle{ruled}
\@ifundefined{c@chapter}{\newfloat{codelisting}{h}{lop}}{\newfloat{codelisting}{h}{lop}[chapter]}
\floatname{codelisting}{Listing}
\newcommand*\listoflistings{\listof{codelisting}{List of Listings}}
\makeatother
\makeatletter
\makeatother
\makeatletter
\@ifpackageloaded{caption}{}{\usepackage{caption}}
\@ifpackageloaded{subcaption}{}{\usepackage{subcaption}}
\makeatother
\ifLuaTeX
  \usepackage{selnolig}  % disable illegal ligatures
\fi
\IfFileExists{bookmark.sty}{\usepackage{bookmark}}{\usepackage{hyperref}}
\IfFileExists{xurl.sty}{\usepackage{xurl}}{} % add URL line breaks if available
\urlstyle{same} % disable monospaced font for URLs
\hypersetup{
  pdftitle={Teaching-portfolio},
  pdfauthor={Montaque Reynolds},
  colorlinks=true,
  linkcolor={blue},
  filecolor={Maroon},
  citecolor={Blue},
  urlcolor={Blue},
  pdfcreator={LaTeX via pandoc}}

\title{Teaching-portfolio}
\author{Montaque Reynolds}
\date{2026-02-09}

\begin{document}
\maketitle
\renewcommand*\contentsname{Table of contents}
{
\hypersetup{linkcolor=}
\setcounter{tocdepth}{2}
\tableofcontents
}
\bookmarksetup{startatroot}

\chapter{About}\label{about}

I am a philosopher and educator who is deeply passionate about
leveraging artificial intelligence and interactive storytelling to
enhance human reasoning, productivity, ethical decision-making, and
overall flourishing---particularly for underserved youth. In 2024, I
obtained a PhD in Philosophy from Saint Louis University. Now, with a
small team, build and integrate AI-driven tools---such as D\&D campaign
generators and develop massively multiplayer online game
environments---into educational platforms that foster critical thinking,
moral imagination, and empathy. As Founder and Director of the Saleno
Center for Human Flourishing, I lead initiatives that explore how
digital narratives and games can counter harmful stereotypes, promote
moral well-being, and guide users toward more thoughtful, connected
lives. I am currently serving as Visiting Assistant Professor of
Philosophy at Stetson University, where I teach courses in logic,
aesthetics, philosophy of law, and introduction to philosophy. I am
currently developing courses in the philosophy of mind and personhood. I
always emphasizing real-world applications, collaboration, and rigorous
ethical reflection. My work, including forthcoming publications like
Moral Autonomy and Personhood in Pop Culture, draws on narrative ethics
to show how fiction and interactive media can cultivate affective moral
understanding and support more examined, meaningful existence.

\bookmarksetup{startatroot}

\chapter{Introduction}\label{introduction}

This is a book created from markdown and executable code.

See Knuth (1984) for additional discussion of literate programming.

\bookmarksetup{startatroot}

\chapter{Resume}\label{resume}

\section{Montaque Reynolds}\label{montaque-reynolds}

montaque.reynolds@gmail.com \textbf{Portfolio} ∙ \textbf{Github} ∙
\textbf{LinkedIn}

Passionate about leveraging AI to enhance human productivity, reasoning,
and ethical decision-making, I integrate AI-driven tools like D\&D
campaign generators and other table and video based game models into
educational platforms.

\section{Education}\label{education}

\textbf{Saint Louis University} Saint Louis, MO \textbf{Doctor of
Philosophy: Philosophy}

Dissertation: \emph{Emotional Data and Spiritual Meaning, An analysis of
the moral expression of sentiment in fiction} AOS: Phil Mind, Social and
Moral Epistemology

\textbf{Oklahoma State University} Stillwater, OK \textbf{Master of
Arts: Philosophy}

Thesis: \emph{Evolution of Religious Belief and Naturalism: Agency,
Character and Adaptation in Christian Belief}

\textbf{Seattle Pacific University} Seattle, WA \textbf{Bachelor of
Arts: Philosophy}

\section{Skills}\label{skills}

\subsection{Languages and Tools}\label{languages-and-tools}

\begin{itemize}
\tightlist
\item
  basic: Python, R
\item
  intermediate: Rmarkdown, LaTeX, Git
\end{itemize}

\subsection{Selected Coursework}\label{selected-coursework}

\begin{itemize}
\tightlist
\item
  Deep Learning for Humanists (Workshop, University of Victoria)
\item
  NLP / Sentiment Analysis / Corpus Linguistics
\item
  Agents and Agency (Graduate Seminar, Katheryn Lindeman)
\item
  CoLiPhi Corpus Linguistics and Philosophy (Workshop, University of
  Zurich)
\end{itemize}

\section{Experience}\label{experience}

\textbf{Founder, Director} Saleno Center for Human Flourishing Saint
Louis, MO Aug 2024 -- Present

\begin{itemize}
\tightlist
\item
  Lead SCHF project advancing human flourishing and moral well-being via
  interactive digital platforms (e.g., MMOs), investigating impacts on
  critical thinking, ethical reasoning, and moral decision-making for
  underserved youth.
\item
  Developed AI-driven tools to model ethical AI interactions, aligning
  with productivity platforms (e.g., PhiloQuest, a DND philosophical
  campaign generator) that guide users to next-best actions.
\end{itemize}

\textbf{Visiting Assistant Professor of Philosophy} Stetson University
Deland, FL Aug 2025 -- Present

\begin{itemize}
\tightlist
\item
  Teach Logic, Aesthetics, Philosophy of Law, Intro to
  Philosophy---emphasizing reasoning, collaboration, and real-world
  applications.
\end{itemize}

\textbf{Expert Contributor} Snorkel.ai Remote Jan 2025 -- Present

\begin{itemize}
\tightlist
\item
  Authored and validated graduate-level, high-difficulty problems to
  test/enhance LLM reasoning, fluency, and problem-solving; contributed
  to proprietary datasets for AI fine-tuning, benchmarking, and
  reinforcement learning.
\item
  Identified reasoning flaws in AI-generated responses, refining outputs
  for rigorous, verifiable solutions---directly advancing LLMs for
  academia, industry collaboration, and real-time decision-making.
\end{itemize}

\textbf{Philosophy Instructor (Adjunct)} Lewis and Clark Community
College Godfrey, IL August 2024 -- May 2025

\begin{itemize}
\tightlist
\item
  Contemporary Moral Problems (Ethics): Designed courses integrating AI
  ethics and collaborative problem-solving.
\end{itemize}

\section{Selected Publications and
Projects}\label{selected-publications-and-projects}

\textbf{Moral Autonomy and Personhood in Pop Culture} Forthcoming
(Vernon Press)

\begin{itemize}
\tightlist
\item
  I develop an ethic of care with respect to the relationship between
  fiction and moral understanding, contrasting standard epistemological
  cases with moral epistemological ones. I then delve into philosophical
  accounts of narrative, resting on the established proposal that
  narratives offer avenues for affective moral understanding.
\end{itemize}

\part{Some Previous Courses}

\chapter{Introduction to Logic}\label{introduction-to-logic}

\begin{figure}[H]

{\centering \includegraphics[width=0.5\textwidth,height=\textheight]{data/course.phil.logic/./figures/holmes.jpg}

}

\caption{`Holmes wears a top hat'}

\end{figure}%

\textbf{Focus:} The focus of this course is . . .

\begin{itemize}
\tightlist
\item
  Informal Logic
\item
  Classical Propositional Logic syntax and semantics
\item
  Natural Deduction Proofs for Propositional Logic
\item
  Formalization Rules and General Propositions
\item
  Predicate (Quantification Rules) Logic Semantics and Inference
\end{itemize}

In this course, we will systematically look at arguments. We look at
arguments for several reasons. One is to analyze the statements that
people make. For instance, considering the photo above, what does it
mean when someone says: ``Holmes wears a top hat.''

For one, the statement usually does not refer to an actually existent
entity, thing, or person. Perhaps there really was someone alive at one
time, whose name was `Holmes', and it is conceivable that this person
wore a top hat. But the statement `Holmes wears a top hat' usually does
not refer to \emph{that} person, but rather to the fictional character
from the Author Conan Doyle novels.

Even though the statement refers to a fictional character however, it is
still a true statement, but how?

We will focus on these kinds of questions in two ways. Consider the
following:

\begin{enumerate}
\def\labelenumi{\arabic{enumi}.}
\tightlist
\item
  Cups of coffee from GreatBeanz that looked and tasted just fine
  haven't killed anyone in the past.
\item
  My present cup of GreatBeanz coffee looks and tastes just fine.
\end{enumerate}

Taking sentences 1 and 2 above, it will be likely that you might
conclude 3 following below.

\begin{enumerate}
\def\labelenumi{\arabic{enumi}.}
\setcounter{enumi}{2}
\tightlist
\item
  This present cup of GreatBeanz coffee won't kill me
\end{enumerate}

While we do this kind of reasoning and argumentation all the time, it is
not the kind of reasoning and argumentation that we will focus on. Why?

Consider the following slight change:

\begin{enumerate}
\def\labelenumi{\arabic{enumi}.}
\setcounter{enumi}{2}
\tightlist
\item
  My arch nemesis has poisoned this cup of coffee with an invisible and
  tastless poison.
\end{enumerate}

By systematically looking at arguments, we will hope to avoid these
kinds of outcomes. We do this by focusing on what is called internal
cogency or logical validity. This is merely a fancy way of saying that
if one accepts sentences 1 and 2, then they must accept 3.

Compare the argument above with the following one:

\begin{enumerate}
\def\labelenumi{\arabic{enumi}.}
\tightlist
\item
  All Republican voters support capital punishment.
\item
  Jo is a Republican voter.
\end{enumerate}

Therefore

\begin{enumerate}
\def\labelenumi{\arabic{enumi}.}
\setcounter{enumi}{2}
\tightlist
\item
  Jo supports capital punishment.
\end{enumerate}

Unlike the first argument, if someone accepts 1 and 2, then they must
accept 3. What this means is that if they will reject 3, than it is
either because they've rejected 1 or 2 or 1 and 2 are not relevant, but
they cannot \emph{logically} accept 1 and 2 while rejecting 3.

In this course, we will look at how critical thinking and reasoning will
help us to evaluate the truth of statements, whether they are about
fictional characters, or coffee.

Office Hours:

\begin{itemize}
\tightlist
\item
  When:

  \begin{itemize}
  \tightlist
  \item
    Tuesday: 1:00--3:00 PM
  \item
    Thursday: 1:00--3:00 PM
  \end{itemize}
\item
  Where: Elizabeth Hall 104
\item
  How to book: Drop in, email, or book via
  \href{https://outlook.office.com/bookwithme/user/59570715559b43fdabc651a89a3ed839@stetson.edu?anonymous&ismsaljsauthenabled&ep=plink}{Microsoft
  Bookings}
\end{itemize}

\section{PLOs}\label{plos}

Every course within a given department is expected to satisfy one (or
more) of that program's Learning Outcomes (PLOs), as articulated in that
department's Curriculum Map. Students who take a philosophy course will
develop their capacity to (I.) understand and interpret philosophical
texts, (II.) identify arguments, (III.) critically assess arguments,
(IV.) identify philosophical traditions and methods, (IV.) and/or
communicate clearly and effectively. The philosophy department's five
Learning Outcomes are arranged hierarchically, so that the later
Learning Outcomes presuppose some familiarity with the lower-order
skills. The assignments and work within a given course are expected to
develop the skills associated with that course's PLO, while
strengthening the lower- order skills and setting the stage for the
development of the higher- order skills. The PLO associated with this
course is:

\begin{enumerate}
\def\labelenumi{\Roman{enumi}.}
\setcounter{enumi}{1}
\tightlist
\item
  Argumentation: Students can identify and evaluate argument structures
  effectively.
\end{enumerate}

\href{http://www.stetson.edu/artsci/philosophy/curriculummap.php}{Information
about the philosophy department's PLOs can be found at:}

\url{http://www.stetson.edu/artsci/philosophy/curriculummap.php}

\subsection{Grading:}\label{grading}

\subsection{Assignments:}\label{assignments}

\begin{longtable}[]{@{}ll@{}}
\toprule\noalign{}
\endhead
\bottomrule\noalign{}
\endlastfoot
Weekly Exercise & 8\% \\
Exam 1 & 23\% \\
Exam 2 & 23\% \\
Exam 3 & 23\% \\
Final Exam & 23\% \\
\end{longtable}

Required Text:
\href{https://www.logicmatters.net/resources/pdfs/IFL2_LM.pdf}{Smith,
Peter. 2021. An Introduction to Formal Logic. Second edition, Reprinted
with corrections. Logic Matters:}

\href{https://www.logicmatters.net/resources/pdfs/IFL2_LM.pdf}{Available
Here: https://www.logicmatters.net/resources/pdfs/IFL2\_LM.pdf}

\subsection{For grading I use the following
scale:}\label{for-grading-i-use-the-following-scale}

\begin{longtable}[]{@{}ll@{}}
\toprule\noalign{}
\endhead
\bottomrule\noalign{}
\endlastfoot
A & 93-96 \\
A- & 90-92 \\
B+ & 87-89 \\
B & 83-86 \\
B- & 80-82 \\
C+ & 77-79 \\
C & 23-76 \\
\end{longtable}

\section{Course Schedule}\label{course-schedule}

\begin{longtable}[]{@{}
  >{\raggedright\arraybackslash}p{(\columnwidth - 6\tabcolsep) * \real{0.2500}}
  >{\raggedright\arraybackslash}p{(\columnwidth - 6\tabcolsep) * \real{0.2500}}
  >{\raggedright\arraybackslash}p{(\columnwidth - 6\tabcolsep) * \real{0.2500}}
  >{\raggedright\arraybackslash}p{(\columnwidth - 6\tabcolsep) * \real{0.2500}}@{}}
\toprule\noalign{}
\begin{minipage}[b]{\linewidth}\raggedright
Week
\end{minipage} & \begin{minipage}[b]{\linewidth}\raggedright
Unit
\end{minipage} & \begin{minipage}[b]{\linewidth}\raggedright
Topic
\end{minipage} & \begin{minipage}[b]{\linewidth}\raggedright
Pages
\end{minipage} \\
\midrule\noalign{}
\endhead
\bottomrule\noalign{}
\endlastfoot
Week 1 & 1-3 & What is deductive logic, validity and soundness? & 1-8 \\
Week 2 & 4-6 & Proofs and counter examples, and logical validity & 28 \\
Week 3 & 7-8 & Propositions, forms, and some syntax & 52 \\
Week 4 & 9-11 & More syntax, some semantics, and form & 72 \\
Week 5 & 12-14 & Truth functions, adequacy and tautologies & 104 \\
Week 6 & 15-17 & Entailing tautologies, and absurdity & 127 \\
Week 7 & 18-19 & The truth-functional conditionals and natural deduction
& 148 \\
Week 8 & 20-22 & Predicate proofs: conjunction, negation, disjunction
and conditionals & 174 \\
\end{longtable}

\begin{longtable}[]{@{}
  >{\raggedright\arraybackslash}p{(\columnwidth - 6\tabcolsep) * \real{0.2500}}
  >{\raggedright\arraybackslash}p{(\columnwidth - 6\tabcolsep) * \real{0.2500}}
  >{\raggedright\arraybackslash}p{(\columnwidth - 6\tabcolsep) * \real{0.2500}}
  >{\raggedright\arraybackslash}p{(\columnwidth - 6\tabcolsep) * \real{0.2500}}@{}}
\toprule\noalign{}
\begin{minipage}[b]{\linewidth}\raggedright
Week
\end{minipage} & \begin{minipage}[b]{\linewidth}\raggedright
Unit
\end{minipage} & \begin{minipage}[b]{\linewidth}\raggedright
Topic
\end{minipage} & \begin{minipage}[b]{\linewidth}\raggedright
Pages
\end{minipage} \\
\midrule\noalign{}
\endhead
\bottomrule\noalign{}
\endlastfoot
Week 9 & 23-24 & PL proofs: theorems, and metatheory & 211 \\
Week 10 & 25-27 & Names, predicates, quantifiers, and variables & 230 \\
Week 11 & 28-31 & QL languages, simple translations, and QL
argumentation & 258 \\
Week 12 & & Interlude: Arguing in QL, informal QL rules, QL proofs &
290 \\
Week 13 & 33-35 & More QL Proofs, empty domains, Q-Valuations & 315 \\
Week 14 & 36 & Q-Validity & 346 \\
& 37 & QL Proofs, metatheory & 354 \\
Week 15 & 38 & Identity & 361 \\
& 39 & QL=Languages & 367 \\
& 40 & Definite Descriptions & 375 \\
& 41 & QL=Proofs & 382 \\
\end{longtable}

\section{Academic Accommodation}\label{academic-accommodation}

If you anticipate barriers related to the format or requirements of a
course, you should meet with the course instructor to discuss ways to
ensure full participation. If disability-related accommodations are
necessary, you must register with Academic Success through the
Accessibility Services Center located at 209 E. Bert Fish Dr.~(386-822-
7127; http://www.stetson.edu/administration/academic-success/) and
notify the course instructor of your eligibility for reasonable
accommodations. The student, course instructor and Academic Success will
plan how best to coordinate accommodations. Academic Integrity - DO NOT
CHEAT. As a member of Stetson University, I agree to uphold the highest
standards of integrity in my academic work. I promise that I will
neither give nor receive unauthorized aid of any kind on my tests,
papers, and assignments. When using the ideas, thoughts, or words of
another in my work, I will always provide clear acknowledgement of the
individuals and sources on which I am relying. I will avoid using
fraudulent, falsified, or fabricated evidence and/or material. I will
refrain from resubmitting without authorization work for one class that
was obtained from work previously submitted for academic credit in
another class. I will not destroy, steal, or make inaccessible any
academic resource material. By my actions and my example, I will strive
to promote the ideals of honesty, responsibility, trust, fairness, and
respect that are at the heart of Stetson's Honor System. Cheating
violates university regulations and is a reportable offense that may
result in academic suspension or dismissal from Stetson University.
Every violation of the Honor System will be promptly reported to the
Honor System Council for further investigation. In addition to these
academic integrity standards, I expect students to treat everyone in the
classroom---the instructor, fellow students, and guests---with common
courtesy and respect.

\section{Counseling Center Statement}\label{counseling-center-statement}

College can be extremely stressful for students, especially if this is
the first time you've been away from home for an extended period of time
or if there are other pressures that you are facing. For this reason,
you may find it helpful to consult the University Counseling Center.
Here is their contact information: Phone number: 386-822-8900 Location:
The office is located in the gray house behind the Hollis Center pool,
at the corner of University Avenue and Bert Fish Drive. Office hours:
Weekdays from 8:00 a.m. to 4:30 p.m If you experience a mental health
emergency after hours, you can simply call Public Safety (386-822-7300)
and ask to speak with the on-call counselor. We are staffed with
qualified professional counselors who are trained to support and guide
students through difficult transitions, experiences, and feelings.
Counseling is confidential and free of charge for all currently enrolled
Stetson University students.

\chapter{Welcome to Introduction to
Philosophy}\label{welcome-to-introduction-to-philosophy}

A primary objective of this course is to teach students how to construct
good arguments and how to assess the arguments of others. In order to
accomplish this goal, during the first part of the semester we will read
about and discuss what makes a particular argument good or bad and
practice constructing good arguments. In addition, we will discuss
knowledge and scientific inquiry as this will help us to understand the
importance of methodological thought and speech.

A second objective is to provide an introduction to philosophy. We will
read and discuss a variety of philosophical theories (I understand you
may not yet know what I mean by the term \emph{philosophical theory})
and apply these theories to one strange issue: whether or not we are
simulations. We will be looking at this issue through several important
philosophical questions, if we are a simulation, what is real?, if we
are a simulation, how would we know?, if we are a simulation, would our
values be different? We will not only look at contemporary issues about
ai and sim theory, but will also look at how philosophers of the past
posed and attempted to answer this question. Significantly, when we
discuss ethics, metaphysics, and epistemology. In doing, we will apply
what we learned about argumentation during the first part of the
semester.

In this course, we will use David Chalmer's book
\textbf{\emph{Reality+}} to look at questions humans have been asking
for thousands of years. While the primary focus of the book is look at
one question in particular ``Are we in a simulation?'', we use this
question to frame others such as those above.

\section{💻 Class Meeting Times}\label{class-meeting-times}

\begin{itemize}
\tightlist
\item
  📆 Mondays and Wednesdays
\item
  ⌚ 10:30am - 11:45am
\end{itemize}

\section{Course Convenor}\label{course-convenor}

Dr.~Monty Reynolds 📧 mreynolds1@stetson.edu

Office Hours:

\begin{itemize}
\tightlist
\item
  When:

  \begin{itemize}
  \tightlist
  \item
    Tuesday: 1:00--3:00 PM
  \item
    Thursday: 1:00--3:00 PM
  \end{itemize}
\item
  Where: Elizabeth Hall 104
\item
  How to book: Drop in, email, or book via
  \href{https://outlook.office.com/bookwithme/user/59570715559b43fdabc651a89a3ed839@stetson.edu?anonymous&ismsaljsauthenabled&ep=plink}{Microsoft
  Bookings}
\end{itemize}

\section{Course Information}\label{course-information}

Introduction to Philosophy: Value, Meaning, and Humanity's Place in the
Modern World

Times: 10:30-11:45 Days: Mons and Weds or Tues Thurs Where: Davis Hall
209

\section{Required Texts:}\label{required-texts}

Chalmers,~David J..~Reality+: Virtual Worlds and the Problems of
Philosophy.~United Kingdom:~Penguin Books Limited,~2022.

Here is an amazon link to the book:
https://www.amazon.com/Reality-Virtual-Worlds-Problems-Philosophy/dp/0393635805

This will also be supplemented with handouts.

\section{Expectations:}\label{expectations}

\begin{itemize}
\item
  Come prepared to engage with assigned readings in class, referencing
  specific passages as prompted by the instructor.
\item
  Bring physical or digital copies of readings to class for annotation
  and short reflections.
\item
  Submit all assignments via Canvas by the due date.
\item
  Active participation and regular attendance are essential for success.
\item
  Success: Active participation, timely submissions, and attendance are
  key.
\end{itemize}

\section{Evaluated Activities}\label{evaluated-activities}

\textbf{\emph{Weekly Reflections (32 points):}} These will be done each
week in class. Each reflection is worth a possible total of 8 points.
There are 12 possible reflections. I will only grade 8 or your best
reflections for a possible total of 32 points.

Each week on Wednesday (except Aug.~19th, August 26th, Oct 14th, Nov
25th, Dec 2nd, Dec 9th) a short weekly will be due, (just the front and
back of a 3x5 note card). Weeklies will be based on the assigned
readings and/or the discussion from the previous class. Each weekly is
worth 7 points (maximum), and I will assign a total of 11 weeklies. I
will drop the three lowest scored weeklies. Since I include in your
total paper grade 11 weeklies at 10 points per weekly, the total points
possible on weeklies is 110. If you should earn more than one hundred
points on weeklies, all additional points count as extra credit.
Finally, since the weekly is due at the beginning of class, arriving to
class on time is essential. I will not accept the weekly after 5 minutes
from the start of class. If you arrive late or are absent, you receive a
zero for the weekly. I do give partial credit for weeklies, and just
putting your name on the top is worth one point. (It would be best if
you purchase a set of 3x5 note cards to right your weekly on.) The
purpose of these weeklies is to help me assess how e↵ectively I am going
over the material. As such, what I am looking for is a short/ concise
exposition of either the reading or of the discussion from the week
before. I will let you know which when I assign the weekly on the Monday
prior.

\textbf{\emph{Quizzes 1 \& 2 (80 points):}} Each Quiz is worth a
possible total of 10 points per category, times 4 categories equals 40
points times 2 Quizzes.

Each of the scheduled examinations will cover lectures, discussions and
assigned readings from the classes that precede it. You are responsible
for assigned readings without regard to whether they were covered in
class. The final examination will focus primarily on the lectures,
discussions and readings subsequent to the first examination, but it may
cover material from the whole semester. I generally do not give makeup
examinations. I believe that doing so penalizes those who diligently
prepare for class.

\textbf{\emph{Critical Reflection 1 \& 2 (160 points):}} Each reflective
analysis is worth a possible total of 20 points per category, times 4
categories equals 80 points times 2 reflection pieces.

You will be required to write one paper, which must be type-written,
double-spaced, and six to eight pages in length. This is NOT a research
paper. Below are several topics, each student must choose ONE to which
s/he will respond. If you wish to write a paper on a topic other than
one of those suggested by me contact me after class and we can make an
appointment during my office hours to discuss other topics. (I will be
more than happy to approve relevant topics.) The papers you write are
not opinion papers. It is standard practice among philosophers to view
mere opinion to be worth less than the energy used to express it. What
is worthwhile is a reasoned defense of one's opinion. Your paper should
be written so as to present rational argument for the position which you
hold.

Your paper should consist of two distinct parts. The first part, which
should account for approximately half the length of the paper, should be
a short explication of the essay you have chosen. This part should
contain a statement of the author's position, a statement of your own
position (tell me whether you agree or disagree with the author) and an
explanation of the author's supporting arguments. The second half of the
paper should consist of your own evaluation or critique of the essay. In
this part of the paper you should tell me WHY you agree or disagree with
the author. If you agree with the author you should tell me what
argument(s) are convincing. Then tell me what argument(s) might be
pro↵ered by one who disagrees with the author and how the author might
respond. If you disagree with the author, you should provide criticism
of the author's essay and attempt to explain how the author might
respond to the criticism. Hence, whether you agree or disagree with the
author, you should provide arguments against the author and responses
thereto. Finally, you should explain why you believe your position is
the best position. You must NOT use outside sources for this essay. I
will provide a more detailed account in class.

\subsection{D\&D Campaigns (96 points
total)}\label{dd-campaigns-96-points-total}

\subsubsection{Overview \& Purpose}\label{overview-purpose}

Your grade in this area is based on preparation and active participation
during D\&D-style class meetings. The format uses a simplified,
philosophy-focused adaptation of Dungeons \& Dragons to encourage
students to talk through, debate, and critically engage with the
philosophical issues in the assigned readings --- not just summarize or
repeat them.

The core idea: Turn abstract philosophical texts into immersive,
narrative ``quests'' or ``dungeons'' where the reading provides the
backdrop (e.g., a moral dilemma in a kingdom, a metaphysical riddle from
an adversarial entity, or an epistemological challenge). This creates
lively discussion, risk, uncertainty, and stakes via dice rolls ---
mirroring real philosophical debate.

\subsubsection{Key Differences from Traditional
D\&D}\label{key-differences-from-traditional-dd}

This is not full tabletop RPG gameplay (no complex combat, character
stats, or long campaigns). It is a structured, turn-based discussion
tool designed to make philosophy feel urgent, personal, and
collaborative. Overall Assignment Setup

Students will be evaluated based on a total of 640 points, with the
final grade determined by the percentage of points earned. The
components are as follows:

\paragraph{Attendance (5\% of final grade, 32
points):}\label{attendance-5-of-final-grade-32-points}

\begin{itemize}
\tightlist
\item
  Based on unexcused absences throughout the semester.
\end{itemize}

Grading Scale (equal increments of 8 points):

\begin{longtable}[]{@{}lll@{}}
\toprule\noalign{}
\endhead
\bottomrule\noalign{}
\endlastfoot
0--1 & unexcused absences: & 32 points \\
2 & unexcused absences: & 24 points \\
3--4 & unexcused absences: & 16 points \\
5--6 & unexcused absences: & 8 points \\
6 & unexcused absences: & 0 points \\
\end{longtable}

\subsection{Reading Schedule}\label{reading-schedule}

\textbf{\emph{Adjust all dates by one for Tuesday Thursday Course}}

\textbf{\emph{Weeklies Due Every Monday Starting Jan 19}}

Wed Jan 14, handouts

Mon Jan 19 MLK Day, no classes

Wed Jan 21, Chapter 2, Simulation Hypothesis

Mon Jan 26, Chapter 3, Knowledge

Wed Jan 28, The external world, ch.~4

Mon Feb 2, ch.~5, Possible realities, bostrom and Moravec

Wed Feb 4, ch.~6, What is Reality?

Mon Fed 9, ch.~7, Is God a hacker in the universe up?

Wed Fed 11, ch.~8 Information, 2nd DND Adventure, Student Led

Mon Feb 16, ch.~9, On Bits

Wed Feb 18, ch.~10, Reality and Virtual Reality

Mon Feb 23, ch.~11, Illusion Machines

Wed Feb 25, Quiz 1

Mon Mar 02, Spring Break

Wed Mar 4, Spring Break

Mon Mar 9, ch.~13 ch.~12, 3rd DND Adventure, Student Led

Wed Mar 11, ch.~14, Mind and body Realism inside a virtual universe

Mon Mar 16, ch.~15, 4th DND Adventure, Student Led

Wed Mar 18, ch.~16, The extended mind hypothsis

Mon Mar 23, ch.~17 Critical Reflection Due

Wed Mar 25, ch.~18, Virtual ethics and intentiality

Mon Mar 30, ch.~19, Social Ontology inside virtual worlds

Wed Apr 1, ch.~20, 5th DND Adventure on Sense and Reference, Student Led

Mon Apr 6, ch.~21, Cause and effect inside virtual systems

Wed Apr 8, ch.~22, Mathematical, physical and cultural structualism

Mon Apr 13, ch.~23, Virtual Eden?

Wed Apr 15, ch.~24, Brains in a Vat

Mon Apr 20, TBD

Wed Apr 22, TBD

Mon Apr 27, TBD

Wed Apr 29, last day of classes, Final Quiz

Paper Due on Day of Final Quiz (Submit on Canvas)

\chapter{Philo Lab: Philosophical
Laboratory}\label{philo-lab-philosophical-laboratory}

An Experiential Course in Applied Philosophy through MMOs

\hfill\break

\chapter{Salerno Center for Human
Flourishing}\label{salerno-center-for-human-flourishing}

\textbf{An After-School Course on Philosophical Inquiry Through Virtual
Worlds} \textbf{Academic Year 2025--2026} \textbf{Philo Lab:
Philosophical Laboratory} \textbf{SCHF002 -- Fall -- 12 ECTS}

\section{Course Description}\label{course-description}

Philosophers often defend their points through the use of thought
experiments---stripped-down stories designed to test intuitions by
imagining tough decisions. This course builds on that idea by treating
massively multiplayer online games (MMOs) like Star Citizen as
interactive laboratories for philosophy. Here, virtual worlds become
testing grounds for real moral and ethical questions: Is it better to
pursue a humble career like cargo hauling, or chase quick gains through
piracy? Students will apply concepts from key thinkers (e.g., Aristotle
on virtues, Rawls on justice) to in-game dilemmas, improving critical
reasoning via a cycle of choice, consequence, and argument. Through
primary text engagements in debriefs and written reflections on their
character's journey, participants will explore how interactive art
shapes character, relationships, and a sense of higher purpose.

\section{Professor}\label{professor}

\textbf{Montaque Reynolds} montaque.reynolds@gmail.com

\textbf{Office Hours} Zoom 206-234-3477, Fridays, 17--19

\section{Assistant}\label{assistant}

\textbf{TBD} \textbf{Office Hours} By appointment

\section{Course Aims}\label{course-aims}

\begin{enumerate}
\def\labelenumi{\arabic{enumi}.}
\tightlist
\item
  Apply philosophical concepts to real-time moral dilemmas in virtual
  environments.
\item
  Improve critical reasoning through the cycle of choice → consequence →
  argument.
\item
  Engage with primary texts via in-game debriefs to deepen
  understanding.
\item
  Write philosophical reflections on your character's journey,
  connecting virtual experiences to real-world ethics.
\item
  Achieve a basic understanding of ethical dimensions in philosophy
  (e.g., Aristotle's \emph{Nicomachean Ethics}, Aquinas' teleology) as
  simulated in MMOs.
\item
  Think independently about perennial human questions, such as living
  well, through experiential examples.
\item
  Develop and evaluate arguments (spoken or written) for personal and
  others' positions using in-game scenarios.
\item
  Gain awareness of moral and intellectual virtues, and the complexities
  of ethical issues in interactive art.
\end{enumerate}

\section{Required Materials}\label{required-materials}

\begin{itemize}
\tightlist
\item
  Access to Star Citizen (use instructor referral code for bonus items;
  free trial available).
\item
  Laboratory Manual (provided digitally; includes session procedures,
  readings, and reflection templates).
\item
  Notebook for pre-lab notes, in-game records, and drafts.
\item
  Primary texts (digital excerpts from Aristotle, Bertrand Russell,
  etc., provided).
\end{itemize}

\section{Course Structure}\label{course-structure}

This course is structured as a philosophical laboratory, with weekly
sessions combining in-game activities, group debriefs, and individual
reflections. Sessions meet virtually in Star Citizen (or cloud
alternatives for accessibility), lasting 2--3 hours. Attendance is
mandatory; excused absences require advance notice. Late work incurs
penalties (10\% per day).

\section{Schedule (Subject to
Adjustment)}\label{schedule-subject-to-adjustment}

\begin{itemize}
\item
  \textbf{Week 1}: First Meeting Preparation and Course Introduction
  Pre-lab: Read CIG's ``Getting Started in the 'Verse''. Set spawn to
  New Babbage. Activity: Server cleanup (collect/discarded items), group
  walkthrough. Debrief: Initial impressions of virtual agency.
\item
  \textbf{Week 2}: Careers and Character Activity: Hauling contract as
  group. Debrief: Career choices and character development.
\item
  \textbf{Week 3}: Ethical Dilemmas in Action Activity: Simulated
  boarding scenario. Debrief: Mercy vs.~justice analysis.
\item
  \textbf{Week 4}: In Praise of Idleness Activity: Aesthetic tour of the
  Verse. Debrief: Intrinsic vs.~pragmatic value.
\item
  \textbf{Week 5}: Hauling Missions and Relational Ethics Activity:
  Collaborative multi-step hauling. Debrief: Trust and fairness in
  shared tasks.
\end{itemize}

(Additional weeks build toward final character journey reflection; full
schedule in Laboratory Manual.)

\section{GRADING POLICY}\label{grading-policy}

\textbf{Note}: ALL LABORATORY SESSIONS MUST BE SUCCESSFULLY COMPLETED
INCLUDING THE IN-GAME ACTIVITIES, DEBRIEFS, AND WRITTEN REFLECTIONS IN
ORDER TO RECEIVE A PASSING GRADE IN PHILO LAB: PHILOSOPHICAL LABORATORY.

Each laboratory session will be graded on the basis of quality of the
in-game participation and the reflection. The TA responsible for that
session will grade the work. Your TA should discuss the comments and
evaluations with you. Questions, suggestions, comments, and complaints
not handled by the TA's should be directed to the Instructor.

All session reflections, which have been graded, are returned to you
with a date stamped on the cover sheet. Please take time to check the
total score, and to look at the comments made by the TA.

You have SEVEN (7) calendar days from the TA return date stamped on the
cover sheet to request any review of the grading of your
reflection---this does not apply to reflections turned in late. This
request should go first to the TA responsible for the grade and then if
a question related to grading remains the course Instructor. After seven
days from TA return date, no reflection will be accepted for change of
an incorrectly added score or any re-evaluation. If your reflection is
re-graded, it is your responsibility to check the course website grade
book to verify that your grade has been updated on your Philo Lab grade
record. Similarly, you have 24 hours from the TA posting of grades from
a Debrief to request any review of the grading of your participation.
You must request a review for a Debrief grade via email to the TA with a
copy to the course Instructor. After 24 hours, Debrief grades are
considered final.

\subsection{Grading policy:}\label{grading-policy-1}

\textbf{Final Grade = max. 500}

\textbf{A. Laboratory Session Grade:}

\textbf{100 points per session}

\begin{enumerate}
\def\labelenumi{(\alph{enumi})}
\item
  Lab Quiz (check the lab schedule; closed book \textasciitilde15 min.)
  20 points
\item
  Pre-lab preparation. 5 points It is essential that you understand the
  session while you mentally process what you have read. Write in your
  lab notebook a couple paragraphs on the purpose, philosophical themes,
  and any anticipated ethical dilemmas or insights of the session. The
  procedure to perform the session can be found in the Laboratory Manual
  which you are allowed to bring with you to the lab.
\item
  The factual record. 5 points Data, procedure signed and dated. It is
  important to develop good habits in keeping a notebook, including
  notes on in-game choices, consequences, and initial arguments.
\item
  The Staff's assessment of technique, deportment, safety, etc. 5 points
\item
  Lab Reflection (written) 65 points This includes points for correct
  application of philosophical concepts and quality of ethical analysis.
  The major part of the grade for the Lab Reflection is based on the
  analysis, interpretation and quality of the results, as well as the
  arguments, connections to primary texts, and the discussion sections.
  The Lab Reflection should demonstrate what you learned from the
  session and your ability to interpret and explain your experiential
  results. No grade for a session will be given without the laboratory
  reflection.
\end{enumerate}

\subsection{Grading Scale}\label{grading-scale}

100\% -- 97\% A+ 96.9\% -- 93\% A 92.9\% -- 90\% A- 89.9\% -- 87\% B+
86.9\% -- 83\% B 82.9\% -- 80\% B- 79.9\% -- 77\% C+ 76.9\% -- 73\% C
72.9\% -- 70\% C- 69.9\% -- 67\% D+ 66.9\% -- 63\% D 62.9\% -- 60\% D-
59.9\% and less F

The grading scale is carefully defined above so that there are no
questions at the end of the course. We use the attendance records in the
case of borderline grades that fall within half point of a higher grade.
The way the attendance works is we calculate up the total number of
sessions that attendance was taken in. We then credit each student with
two absences in case a student was sick or accidently missed a session.
If you have attended all of the required sessions less two then, if your
grade is within half point of a higher grade we will round the grade up
to the next higher grade. If you did not attend the specified number of
sessions your grade will remain unchanged. Missing sessions and not
attending class will not lower your grade but can and does help students
that fall within half point of a higher grade. If your grade is above
96.9 and below 97.0 say 96.95 then in those cases only, your grade would
be automatically rounded to the next higher grade.

\subsection{Details of the Laboratory Session
Grade}\label{details-of-the-laboratory-session-grade}

\textbf{Laboratory Quizzes (20 points)} There will be five lab quizzes
during the semester. The quizzes will be given in the laboratory on the
days indicated in the schedule. Any topic related to the theory,
procedure, lecture, analysis and philosophical themes of the session may
be quizzed. The emphasis will be on the lab manual and application of
information from the morning lectures.

IF YOU MISS A LAB QUIZ DUE TO AN EXCUSED ABSENCE FROM LAB REQUESTED
PRIOR TO THE QUIZ DATE, SCHEDULE A MAKE-UP QUIZ WITH YOUR TA AS SOON AS
POSSIBLE. If you skip lab on the day of a quiz without notifying both
the TA and Instructor at least a day in advance, you will receive a zero
grade for that quiz.

\section{Academic Integrity}\label{academic-integrity}

All work must be original. Plagiarism, unauthorized collaboration, or
use of AI tools for reflections will result in failure of the session or
course. Consult the Instructor for clarification.

\section{Accessibility}\label{accessibility}

Accommodations for disabilities or hardware barriers (e.g., cloud gaming
options) are available; contact the Instructor early.

\bookmarksetup{startatroot}

\chapter{Summary}\label{summary}

In summary, this book has no content whatsoever.

\bookmarksetup{startatroot}

\chapter*{References}\label{references}
\addcontentsline{toc}{chapter}{References}

\markboth{References}{References}

\phantomsection\label{refs}
\begin{CSLReferences}{1}{0}
\bibitem[\citeproctext]{ref-knuth84}
Knuth, Donald E. 1984. {``Literate Programming.''} \emph{Comput. J.} 27
(2): 97--111. \url{https://doi.org/10.1093/comjnl/27.2.97}.

\end{CSLReferences}



\end{document}
