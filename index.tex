% Options for packages loaded elsewhere
\PassOptionsToPackage{unicode}{hyperref}
\PassOptionsToPackage{hyphens}{url}
\PassOptionsToPackage{dvipsnames,svgnames,x11names}{xcolor}
%
\documentclass[
  letterpaper,
  DIV=11,
  numbers=noendperiod]{scrreprt}

\usepackage{amsmath,amssymb}
\usepackage{iftex}
\ifPDFTeX
  \usepackage[T1]{fontenc}
  \usepackage[utf8]{inputenc}
  \usepackage{textcomp} % provide euro and other symbols
\else % if luatex or xetex
  \usepackage{unicode-math}
  \defaultfontfeatures{Scale=MatchLowercase}
  \defaultfontfeatures[\rmfamily]{Ligatures=TeX,Scale=1}
\fi
\usepackage{lmodern}
\ifPDFTeX\else  
    % xetex/luatex font selection
\fi
% Use upquote if available, for straight quotes in verbatim environments
\IfFileExists{upquote.sty}{\usepackage{upquote}}{}
\IfFileExists{microtype.sty}{% use microtype if available
  \usepackage[]{microtype}
  \UseMicrotypeSet[protrusion]{basicmath} % disable protrusion for tt fonts
}{}
\makeatletter
\@ifundefined{KOMAClassName}{% if non-KOMA class
  \IfFileExists{parskip.sty}{%
    \usepackage{parskip}
  }{% else
    \setlength{\parindent}{0pt}
    \setlength{\parskip}{6pt plus 2pt minus 1pt}}
}{% if KOMA class
  \KOMAoptions{parskip=half}}
\makeatother
\usepackage{xcolor}
\setlength{\emergencystretch}{3em} % prevent overfull lines
\setcounter{secnumdepth}{5}
% Make \paragraph and \subparagraph free-standing
\ifx\paragraph\undefined\else
  \let\oldparagraph\paragraph
  \renewcommand{\paragraph}[1]{\oldparagraph{#1}\mbox{}}
\fi
\ifx\subparagraph\undefined\else
  \let\oldsubparagraph\subparagraph
  \renewcommand{\subparagraph}[1]{\oldsubparagraph{#1}\mbox{}}
\fi


\providecommand{\tightlist}{%
  \setlength{\itemsep}{0pt}\setlength{\parskip}{0pt}}\usepackage{longtable,booktabs,array}
\usepackage{calc} % for calculating minipage widths
% Correct order of tables after \paragraph or \subparagraph
\usepackage{etoolbox}
\makeatletter
\patchcmd\longtable{\par}{\if@noskipsec\mbox{}\fi\par}{}{}
\makeatother
% Allow footnotes in longtable head/foot
\IfFileExists{footnotehyper.sty}{\usepackage{footnotehyper}}{\usepackage{footnote}}
\makesavenoteenv{longtable}
\usepackage{graphicx}
\makeatletter
\def\maxwidth{\ifdim\Gin@nat@width>\linewidth\linewidth\else\Gin@nat@width\fi}
\def\maxheight{\ifdim\Gin@nat@height>\textheight\textheight\else\Gin@nat@height\fi}
\makeatother
% Scale images if necessary, so that they will not overflow the page
% margins by default, and it is still possible to overwrite the defaults
% using explicit options in \includegraphics[width, height, ...]{}
\setkeys{Gin}{width=\maxwidth,height=\maxheight,keepaspectratio}
% Set default figure placement to htbp
\makeatletter
\def\fps@figure{htbp}
\makeatother
% definitions for citeproc citations
\NewDocumentCommand\citeproctext{}{}
\NewDocumentCommand\citeproc{mm}{%
  \begingroup\def\citeproctext{#2}\cite{#1}\endgroup}
\makeatletter
 % allow citations to break across lines
 \let\@cite@ofmt\@firstofone
 % avoid brackets around text for \cite:
 \def\@biblabel#1{}
 \def\@cite#1#2{{#1\if@tempswa , #2\fi}}
\makeatother
\newlength{\cslhangindent}
\setlength{\cslhangindent}{1.5em}
\newlength{\csllabelwidth}
\setlength{\csllabelwidth}{3em}
\newenvironment{CSLReferences}[2] % #1 hanging-indent, #2 entry-spacing
 {\begin{list}{}{%
  \setlength{\itemindent}{0pt}
  \setlength{\leftmargin}{0pt}
  \setlength{\parsep}{0pt}
  % turn on hanging indent if param 1 is 1
  \ifodd #1
   \setlength{\leftmargin}{\cslhangindent}
   \setlength{\itemindent}{-1\cslhangindent}
  \fi
  % set entry spacing
  \setlength{\itemsep}{#2\baselineskip}}}
 {\end{list}}
\usepackage{calc}
\newcommand{\CSLBlock}[1]{\hfill\break\parbox[t]{\linewidth}{\strut\ignorespaces#1\strut}}
\newcommand{\CSLLeftMargin}[1]{\parbox[t]{\csllabelwidth}{\strut#1\strut}}
\newcommand{\CSLRightInline}[1]{\parbox[t]{\linewidth - \csllabelwidth}{\strut#1\strut}}
\newcommand{\CSLIndent}[1]{\hspace{\cslhangindent}#1}

\KOMAoption{captions}{tableheading}
\makeatletter
\@ifpackageloaded{tcolorbox}{}{\usepackage[skins,breakable]{tcolorbox}}
\@ifpackageloaded{fontawesome5}{}{\usepackage{fontawesome5}}
\definecolor{quarto-callout-color}{HTML}{909090}
\definecolor{quarto-callout-note-color}{HTML}{0758E5}
\definecolor{quarto-callout-important-color}{HTML}{CC1914}
\definecolor{quarto-callout-warning-color}{HTML}{EB9113}
\definecolor{quarto-callout-tip-color}{HTML}{00A047}
\definecolor{quarto-callout-caution-color}{HTML}{FC5300}
\definecolor{quarto-callout-color-frame}{HTML}{acacac}
\definecolor{quarto-callout-note-color-frame}{HTML}{4582ec}
\definecolor{quarto-callout-important-color-frame}{HTML}{d9534f}
\definecolor{quarto-callout-warning-color-frame}{HTML}{f0ad4e}
\definecolor{quarto-callout-tip-color-frame}{HTML}{02b875}
\definecolor{quarto-callout-caution-color-frame}{HTML}{fd7e14}
\makeatother
\makeatletter
\@ifpackageloaded{bookmark}{}{\usepackage{bookmark}}
\makeatother
\makeatletter
\@ifpackageloaded{caption}{}{\usepackage{caption}}
\AtBeginDocument{%
\ifdefined\contentsname
  \renewcommand*\contentsname{Table of contents}
\else
  \newcommand\contentsname{Table of contents}
\fi
\ifdefined\listfigurename
  \renewcommand*\listfigurename{List of Figures}
\else
  \newcommand\listfigurename{List of Figures}
\fi
\ifdefined\listtablename
  \renewcommand*\listtablename{List of Tables}
\else
  \newcommand\listtablename{List of Tables}
\fi
\ifdefined\figurename
  \renewcommand*\figurename{Figure}
\else
  \newcommand\figurename{Figure}
\fi
\ifdefined\tablename
  \renewcommand*\tablename{Table}
\else
  \newcommand\tablename{Table}
\fi
}
\@ifpackageloaded{float}{}{\usepackage{float}}
\floatstyle{ruled}
\@ifundefined{c@chapter}{\newfloat{codelisting}{h}{lop}}{\newfloat{codelisting}{h}{lop}[chapter]}
\floatname{codelisting}{Listing}
\newcommand*\listoflistings{\listof{codelisting}{List of Listings}}
\makeatother
\makeatletter
\makeatother
\makeatletter
\@ifpackageloaded{caption}{}{\usepackage{caption}}
\@ifpackageloaded{subcaption}{}{\usepackage{subcaption}}
\makeatother
\ifLuaTeX
  \usepackage{selnolig}  % disable illegal ligatures
\fi
\IfFileExists{bookmark.sty}{\usepackage{bookmark}}{\usepackage{hyperref}}
\IfFileExists{xurl.sty}{\usepackage{xurl}}{} % add URL line breaks if available
\urlstyle{same} % disable monospaced font for URLs
\hypersetup{
  pdftitle={🗓️ Week 06  Chapter 6: Reality},
  pdfauthor={Introduction to Philosophy by Philip A. Pecorino is licensed under a Creative Commons Attribution-NonCommercial-NoDerivs 3.0 Unported License. here},
  colorlinks=true,
  linkcolor={blue},
  filecolor={Maroon},
  citecolor={Blue},
  urlcolor={Blue},
  pdfcreator={LaTeX via pandoc}}

\title{🗓️ Week 06 Chapter 6: Reality}
\usepackage{etoolbox}
\makeatletter
\providecommand{\subtitle}[1]{% add subtitle to \maketitle
  \apptocmd{\@title}{\par {\large #1 \par}}{}{}
}
\makeatother
\subtitle{PHIL101B Reality}
\author{Montaque Reynolds}
\date{02 September 2026}

\begin{document}
\maketitle
\bookmarksetup{startatroot}

\chapter*{Preface}\label{preface}
\addcontentsline{toc}{chapter}{Preface}

\markboth{Preface}{Preface}

This is a Quarto book.

To learn more about Quarto books visit
\url{https://quarto.org/docs/books}.

\bookmarksetup{startatroot}

\chapter{Introduction}\label{introduction}

This is a book created from markdown and executable code.

See Knuth (1984) for additional discussion of literate programming.

\part{Introduction to Logic}

\chapter{Introduction}\label{introduction-1}

Previously, this course used John Nolt's \emph{Logics} as its text book.
While I still believe that the heavy emphasis on formal logic is
important, it was not intuitive, and logic done well is intuitive. I
eventually discovered Peter (\textbf{smit21b?})'s
\href{https://www.logicmatters.net/}{\emph{Logic Matters}}. I
recommended it to students as a supplementary reading and was pleased
with the feedback and the improvement in their understanding. I am now
relying on Smith's introductory text as the main text for this course,
although I will contrast it with Nolt's more formal approach at various
points.

\textbf{Focus:} The focus of this course is . . .

\begin{itemize}
\tightlist
\item
  Informal Logic
\item
  Classical Propositional Logic syntax and semantics
\item
  Natural Deduction Proofs for Propositional Logic
\item
  Formalization Rules and General Propositions
\item
  Predicate (Quantification Rules) Logic Semantics and Inference
\end{itemize}

\textbf{Free! Course Text:}
\href{https://www.logicmatters.net/resources/pdfs/IFL2_LM.pdf}{Smith,
Peter. 2021. An Introduction to Formal Logic. Second edition, Reprinted
with corrections. Logic Matters.}

\section{Summary}\label{summary}

In the image below, Sherlock Holmes is wearing a top hat. Therefore in
Fiction \emph{f}, it is true that:

\begin{itemize}
\tightlist
\item
  Holmes lived in Baker Street.
\item
  Holmes lived nearer to Paddington Station than to Waterloo Station.
\item
  Holmes was just a person, a person of flesh and blood.
\item
  Holmes really existed.
\item
  Someone lived for many years at 22 iB Baker Street.
\end{itemize}

\begin{figure}[H]

{\centering \includegraphics{data/course.phil.logic/./figures/holmes.jpg}

}

\caption{`Holmes wears a top hat'}

\end{figure}%

\chapter{Syllabus}\label{syllabus}

\begin{figure}[H]

{\centering \includegraphics[width=0.5\textwidth,height=\textheight]{data/course.phil.logic/./figures/holmes.jpg}

}

\caption{`Holmes wears a top hat'}

\end{figure}%

\textbf{Focus:} The focus of this course is . . .

\begin{itemize}
\tightlist
\item
  Informal Logic
\item
  Classical Propositional Logic syntax and semantics
\item
  Natural Deduction Proofs for Propositional Logic
\item
  Formalization Rules and General Propositions
\item
  Predicate (Quantification Rules) Logic Semantics and Inference
\end{itemize}

In this course, we will systematically look at arguments. We look at
arguments for several reasons. One is to analyze the statements that
people make. For instance, considering the photo above, what does it
mean when someone says: ``Holmes wears a top hat.''

For one, the statement usually does not refer to an actually existent
entity, thing, or person. Perhaps there really was someone alive at one
time, whose name was `Holmes', and it is conceivable that this person
wore a top hat. But the statement `Holmes wears a top hat' usually does
not refer to \emph{that} person, but rather to the fictional character
from the Author Conan Doyle novels.

Even though the statement refers to a fictional character however, it is
still a true statement, but how?

We will focus on these kinds of questions in two ways. Consider the
following:

\begin{enumerate}
\def\labelenumi{\arabic{enumi}.}
\tightlist
\item
  Cups of coffee from GreatBeanz that looked and tasted just fine
  haven't killed anyone in the past.
\item
  My present cup of GreatBeanz coffee looks and tastes just fine.
\end{enumerate}

Taking sentences 1 and 2 above, it will be likely that you might
conclude 3 following below.

\begin{enumerate}
\def\labelenumi{\arabic{enumi}.}
\setcounter{enumi}{2}
\tightlist
\item
  This present cup of GreatBeanz coffee won't kill me
\end{enumerate}

While we do this kind of reasoning and argumentation all the time, it is
not the kind of reasoning and argumentation that we will focus on. Why?

Consider the following slight change:

\begin{enumerate}
\def\labelenumi{\arabic{enumi}.}
\setcounter{enumi}{2}
\tightlist
\item
  My arch nemesis has poisoned this cup of coffee with an invisible and
  tastless poison.
\end{enumerate}

By systematically looking at arguments, we will hope to avoid these
kinds of outcomes. We do this by focusing on what is called internal
cogency or logical validity. This is merely a fancy way of saying that
if one accepts sentences 1 and 2, then they must accept 3.

Compare the argument above with the following one:

\begin{enumerate}
\def\labelenumi{\arabic{enumi}.}
\tightlist
\item
  All Republican voters support capital punishment.
\item
  Jo is a Republican voter.
\end{enumerate}

Therefore

\begin{enumerate}
\def\labelenumi{\arabic{enumi}.}
\setcounter{enumi}{2}
\tightlist
\item
  Jo supports capital punishment.
\end{enumerate}

Unlike the first argument, if someone accepts 1 and 2, then they must
accept 3. What this means is that if they will reject 3, than it is
either because they've rejected 1 or 2 or 1 and 2 are not relevant, but
they cannot \emph{logically} accept 1 and 2 while rejecting 3.

In this course, we will look at how critical thinking and reasoning will
help us to evaluate the truth of statements, whether they are about
fictional characters, or coffee.

Office Hours:

\begin{itemize}
\tightlist
\item
  When:

  \begin{itemize}
  \tightlist
  \item
    Tuesday: 1:00--3:00 PM
  \item
    Thursday: 1:00--3:00 PM
  \end{itemize}
\item
  Where: Elizabeth Hall 104
\item
  How to book: Drop in, email, or book via
  \href{https://outlook.office.com/bookwithme/user/59570715559b43fdabc651a89a3ed839@stetson.edu?anonymous&ismsaljsauthenabled&ep=plink}{Microsoft
  Bookings}
\end{itemize}

\section{PLOs}\label{plos}

Every course within a given department is expected to satisfy one (or
more) of that program's Learning Outcomes (PLOs), as articulated in that
department's Curriculum Map. Students who take a philosophy course will
develop their capacity to (I.) understand and interpret philosophical
texts, (II.) identify arguments, (III.) critically assess arguments,
(IV.) identify philosophical traditions and methods, (IV.) and/or
communicate clearly and effectively. The philosophy department's five
Learning Outcomes are arranged hierarchically, so that the later
Learning Outcomes presuppose some familiarity with the lower-order
skills. The assignments and work within a given course are expected to
develop the skills associated with that course's PLO, while
strengthening the lower- order skills and setting the stage for the
development of the higher- order skills. The PLO associated with this
course is:

\begin{enumerate}
\def\labelenumi{\Roman{enumi}.}
\setcounter{enumi}{1}
\tightlist
\item
  Argumentation: Students can identify and evaluate argument structures
  effectively.
\end{enumerate}

\href{http://www.stetson.edu/artsci/philosophy/curriculummap.php}{Information
about the philosophy department's PLOs can be found at:}

\url{http://www.stetson.edu/artsci/philosophy/curriculummap.php}

\subsection{Grading:}\label{grading}

\subsection{Assignments:}\label{assignments}

\begin{longtable}[]{@{}ll@{}}
\toprule\noalign{}
\endhead
\bottomrule\noalign{}
\endlastfoot
Weekly Exercise & 8\% \\
Exam 1 & 23\% \\
Exam 2 & 23\% \\
Exam 3 & 23\% \\
Final Exam & 23\% \\
\end{longtable}

Required Text:
\href{https://www.logicmatters.net/resources/pdfs/IFL2_LM.pdf}{Smith,
Peter. 2021. An Introduction to Formal Logic. Second edition, Reprinted
with corrections. Logic Matters:}

\href{https://www.logicmatters.net/resources/pdfs/IFL2_LM.pdf}{Available
Here: https://www.logicmatters.net/resources/pdfs/IFL2\_LM.pdf}

\subsection{For grading I use the following
scale:}\label{for-grading-i-use-the-following-scale}

\begin{longtable}[]{@{}ll@{}}
\toprule\noalign{}
\endhead
\bottomrule\noalign{}
\endlastfoot
A & 93-96 \\
A- & 90-92 \\
B+ & 87-89 \\
B & 83-86 \\
B- & 80-82 \\
C+ & 77-79 \\
C & 23-76 \\
\end{longtable}

\section{Course Schedule}\label{course-schedule}

\begin{longtable}[]{@{}
  >{\raggedright\arraybackslash}p{(\columnwidth - 6\tabcolsep) * \real{0.2500}}
  >{\raggedright\arraybackslash}p{(\columnwidth - 6\tabcolsep) * \real{0.2500}}
  >{\raggedright\arraybackslash}p{(\columnwidth - 6\tabcolsep) * \real{0.2500}}
  >{\raggedright\arraybackslash}p{(\columnwidth - 6\tabcolsep) * \real{0.2500}}@{}}
\toprule\noalign{}
\begin{minipage}[b]{\linewidth}\raggedright
Week
\end{minipage} & \begin{minipage}[b]{\linewidth}\raggedright
Unit
\end{minipage} & \begin{minipage}[b]{\linewidth}\raggedright
Topic
\end{minipage} & \begin{minipage}[b]{\linewidth}\raggedright
Pages
\end{minipage} \\
\midrule\noalign{}
\endhead
\bottomrule\noalign{}
\endlastfoot
Week 1 & 1-3 & What is deductive logic, validity and soundness? & 1-8 \\
Week 2 & 4-6 & Proofs and counter examples, and logical validity & 28 \\
Week 3 & 7-8 & Propositions, forms, and some syntax & 52 \\
Week 4 & 9-11 & More syntax, some semantics, and form & 72 \\
Week 5 & 12-14 & Truth functions, adequacy and tautologies & 104 \\
Week 6 & 15-17 & Entailing tautologies, and absurdity & 127 \\
Week 7 & 18-19 & The truth-functional conditionals and natural deduction
& 148 \\
Week 8 & 20-22 & Predicate proofs: conjunction, negation, disjunction
and conditionals & 174 \\
\end{longtable}

\begin{longtable}[]{@{}
  >{\raggedright\arraybackslash}p{(\columnwidth - 6\tabcolsep) * \real{0.2500}}
  >{\raggedright\arraybackslash}p{(\columnwidth - 6\tabcolsep) * \real{0.2500}}
  >{\raggedright\arraybackslash}p{(\columnwidth - 6\tabcolsep) * \real{0.2500}}
  >{\raggedright\arraybackslash}p{(\columnwidth - 6\tabcolsep) * \real{0.2500}}@{}}
\toprule\noalign{}
\begin{minipage}[b]{\linewidth}\raggedright
Week
\end{minipage} & \begin{minipage}[b]{\linewidth}\raggedright
Unit
\end{minipage} & \begin{minipage}[b]{\linewidth}\raggedright
Topic
\end{minipage} & \begin{minipage}[b]{\linewidth}\raggedright
Pages
\end{minipage} \\
\midrule\noalign{}
\endhead
\bottomrule\noalign{}
\endlastfoot
Week 9 & 23-24 & PL proofs: theorems, and metatheory & 211 \\
Week 10 & 25-27 & Names, predicates, quantifiers, and variables & 230 \\
Week 11 & 28-31 & QL languages, simple translations, and QL
argumentation & 258 \\
Week 12 & & Interlude: Arguing in QL, informal QL rules, QL proofs &
290 \\
Week 13 & 33-35 & More QL Proofs, empty domains, Q-Valuations & 315 \\
Week 14 & 36 & Q-Validity & 346 \\
& 37 & QL Proofs, metatheory & 354 \\
Week 15 & 38 & Identity & 361 \\
& 39 & QL=Languages & 367 \\
& 40 & Definite Descriptions & 375 \\
& 41 & QL=Proofs & 382 \\
\end{longtable}

\section{Academic Accommodation}\label{academic-accommodation}

If you anticipate barriers related to the format or requirements of a
course, you should meet with the course instructor to discuss ways to
ensure full participation. If disability-related accommodations are
necessary, you must register with Academic Success through the
Accessibility Services Center located at 209 E. Bert Fish Dr.~(386-822-
7127; http://www.stetson.edu/administration/academic-success/) and
notify the course instructor of your eligibility for reasonable
accommodations. The student, course instructor and Academic Success will
plan how best to coordinate accommodations. Academic Integrity - DO NOT
CHEAT. As a member of Stetson University, I agree to uphold the highest
standards of integrity in my academic work. I promise that I will
neither give nor receive unauthorized aid of any kind on my tests,
papers, and assignments. When using the ideas, thoughts, or words of
another in my work, I will always provide clear acknowledgement of the
individuals and sources on which I am relying. I will avoid using
fraudulent, falsified, or fabricated evidence and/or material. I will
refrain from resubmitting without authorization work for one class that
was obtained from work previously submitted for academic credit in
another class. I will not destroy, steal, or make inaccessible any
academic resource material. By my actions and my example, I will strive
to promote the ideals of honesty, responsibility, trust, fairness, and
respect that are at the heart of Stetson's Honor System. Cheating
violates university regulations and is a reportable offense that may
result in academic suspension or dismissal from Stetson University.
Every violation of the Honor System will be promptly reported to the
Honor System Council for further investigation. In addition to these
academic integrity standards, I expect students to treat everyone in the
classroom---the instructor, fellow students, and guests---with common
courtesy and respect.

\section{Counseling Center Statement}\label{counseling-center-statement}

College can be extremely stressful for students, especially if this is
the first time you've been away from home for an extended period of time
or if there are other pressures that you are facing. For this reason,
you may find it helpful to consult the University Counseling Center.
Here is their contact information: Phone number: 386-822-8900 Location:
The office is located in the gray house behind the Hollis Center pool,
at the corner of University Avenue and Bert Fish Drive. Office hours:
Weekdays from 8:00 a.m. to 4:30 p.m If you experience a mental health
emergency after hours, you can simply call Public Safety (386-822-7300)
and ask to speak with the on-call counselor. We are staffed with
qualified professional counselors who are trained to support and guide
students through difficult transitions, experiences, and feelings.
Counseling is confidential and free of charge for all currently enrolled
Stetson University students.

\chapter{Lectures}\label{lectures}

\chapter{Chapter 1: Deductive Logic}\label{chapter-1-deductive-logic}

\begin{longtable}[]{@{}lll@{}}
\toprule\noalign{}
Unit & Topic & Pages \\
\midrule\noalign{}
\endhead
\bottomrule\noalign{}
\endlastfoot
1 & What is deductive logic? & 1-8 \\
2 & Validity and soundness & 9-19 \\
3 & Forms of inference & 20 \\
\end{longtable}

\section{What is an argument?}\label{what-is-an-argument}

\begin{itemize}
\tightlist
\item
  How can we determine the premises of an argument?
\item
  What is an inference marker and what are some examples?
\item
  How do we tell whether a given statement is the premise of an
  argument, or the conclusion?
\end{itemize}

As I mentioned previously, the most important component of this course,
is being able to evaluate arguments for \emph{internal} cogency.

But before we evaluate arguments for \emph{internal} cogency, we must be
able to recognize something as an argument.

\section{Some Examples}\label{some-examples}

We are going to reorganize the following statements into arguments.

\begin{enumerate}
\def\labelenumi{\arabic{enumi}.}
\tightlist
\item
  Most doctors are caring. After all, most ordinary people are caring
  --- and politicians are ordinary people.
\item
  Anyone who knows how to pole vault, even if she doesn't get a gold
  medal in the olympics, will at least get toned arms and abdomen. Jane
  knows how to pole vault, so she at least has toned arms and abdomen.
\item
  Peter is shorter than Paul and Jim is taller than Paul. So Peter is
  shorter than Jim.
\item
  At 12am, Fred is always either bartending, or at home. And assuming
  he's at home, he's building his model train set. Fred was not
  bartending at midnight. So he was building his model train set.
\item
  Kermit is green all over. Hence Kermit is not red all over.
\item
  Every letter is in a pigeon hole. There are more letters than there
  are pigeon holes. So some pigeon hole contains more than one letter.
\item
  Miracles cannot happen. Since, by definition, a miracle is an event
  incompatible with the laws of nature. And everything that happens is
  always consistent with the laws of nature.
\end{enumerate}

There is more than one kind of internal cogency however. We will be
talking about \textbf{\emph{logical validity}}. Internal cogency and
logical validity are what underlie the kinds ``systematic'' evaluations
of arguments that we will learn about in this course. Once again, there
are more than one of these.

Consider the following:

\begin{enumerate}
\def\labelenumi{\arabic{enumi}.}
\tightlist
\item
  Most doctors are caring. After all, most ordinary people are caring
  --- and doctors are ordinary people.

  \begin{enumerate}
  \def\labelenumii{\arabic{enumii}.}
  \tightlist
  \item
    Most ordinary people are caring
  \item
    Doctors are ordinary people
  \item
    Most Doctors are caring
  \end{enumerate}
\item
  Anyone who knows how to pole vault, even if she doesn't get a gold
  medal in the olympics, will at least have toned arms and abdomen. Jane
  knows how to pole vault, so she at least has toned arms and abdomen.

  \begin{enumerate}
  \def\labelenumii{\arabic{enumii}.}
  \tightlist
  \item
    Anyone who knows how to pole vault, even if she doesn't get a gold
    medal in the olympics, will at least get toned arms and abdomen.
  \item
    Jane knows how to pole vault.
  \item
    Jane will at least have toned arms and abdomen.
  \end{enumerate}
\item
  Peter is shorter than Paul and Jim is taller than Paul. So Peter is
  shorter than Jim.

  \begin{enumerate}
  \def\labelenumii{\arabic{enumii}.}
  \tightlist
  \item
    Peter is shorter than Paul and Jim is taller than Paul.
  \item
    So Peter is shorter than Jim.
  \end{enumerate}
\item
  At 12am, Fred is always either bartending, or at home. And assuming
  he's at home, he's building his model train set. Fred was not
  bartending at midnight. So he was building his model train set.

  \begin{enumerate}
  \def\labelenumii{\arabic{enumii}.}
  \tightlist
  \item
    At 12am, Fred is always either bartending, or at home.
  \item
    Assuming he's at home, he's building his model train set.
  \item
    Fred was not bartending at midnight.
  \item
    So Fred was building his model train set at midnight.
  \end{enumerate}
\item
  Kermit is green all over. Hence Kermit is not red all over.

  \begin{enumerate}
  \def\labelenumii{\arabic{enumii}.}
  \tightlist
  \item
    Kermit is green all over.
  \item
    Hence Kermit is not red all over.
  \end{enumerate}
\item
  Every letter is in a pigeon hole. There are more letters than there
  are pigeon holes. So some pigeon hole contains more than one letter.

  \begin{enumerate}
  \def\labelenumii{\arabic{enumii}.}
  \tightlist
  \item
    Every letter is in a pigeon hole.
  \item
    There are more letters than there are pigeon holes.
  \item
    So some pigeon hole contains more than one letter.
  \end{enumerate}
\item
  Miracles cannot happen. Since, by definition, a miracle is an event
  incompatible with the laws of nature. And everything that happens is
  always consistent with the laws of nature.

  \begin{enumerate}
  \def\labelenumii{\arabic{enumii}.}
  \tightlist
  \item
    A miracle is an event that is incompatible with the laws of nature.
  \item
    Everything that happens is always consistent with the laws of
    nature.
  \item
    Therefore miracles cannot happen.
  \end{enumerate}
\end{enumerate}

\section{Kinds of Evaluation}\label{kinds-of-evaluation}

\begin{itemize}
\tightlist
\item
  Are the premises supporting the conclusion true?
\item
  Is the inference from the premises to the conclusion lock-tight?
\end{itemize}

\begin{itemize}
\item
  How do we determine whether the inference step is lock-tight?
\item
  Assume the truth of the premises
\end{itemize}

\section{Deduction and Induction}\label{deduction-and-induction}

\begin{tcolorbox}[enhanced jigsaw, colbacktitle=quarto-callout-important-color!10!white, bottomrule=.15mm, titlerule=0mm, arc=.35mm, colback=white, opacitybacktitle=0.6, rightrule=.15mm, opacityback=0, colframe=quarto-callout-important-color-frame, bottomtitle=1mm, toprule=.15mm, toptitle=1mm, breakable, title=\textcolor{quarto-callout-important-color}{\faExclamation}\hspace{0.5em}{📢 Important}, coltitle=black, left=2mm, leftrule=.75mm]

\begin{quote}
If an inference step from premises to a conclusion is complletely
watertight, i.e.~if the truth of the premises absolutely guarantee the
truth of the conclusion, then we say that this inference step is
\emph{deductively valid}.
\end{quote}

\begin{quote}
Equivalently, when an inference step is deductively valid, we will say
that its premises deductively entail its conclusion. (\textbf{smit94?})
\end{quote}

\end{tcolorbox}

\section{Argument Patterns, or
Generalizing}\label{argument-patterns-or-generalizing}

Lewis Carroll: No interesting poems are unpopular among people of real
taste. No modern poetry is free from affectation. All your poems are on
the subject of soap bubbles. No affected poetry is popular among people
of real taste. Only a modern poem would be on the subject of soap
bubbles. Therefore none of your poems are interesting.

\begin{center}\rule{0.5\linewidth}{0.5pt}\end{center}

\begin{quote}
No interesting poems are unpopular among people of real taste. No modern
poetry is free from affectation. All your poems are on the subject of
soap bubbles. No affected poetry is popular among people of real taste.
Only a modern poem would be on the subject of soap bubbles. Therefore
none of your poems are interesting.
\end{quote}

\begin{itemize}
\item
  \textbf{\emph{Conclusion:}} None of your poems are interesting
\item
  \begin{enumerate}
  \def\labelenumi{(\arabic{enumi})}
  \tightlist
  \item
    No interesting poems are unpopular among people of real taste.
    (premiss)
  \end{enumerate}
\item
  \begin{enumerate}
  \def\labelenumi{(\arabic{enumi})}
  \setcounter{enumi}{1}
  \tightlist
  \item
    No modern poetry is free from affectation. (premiss)
  \end{enumerate}
\item
  \begin{enumerate}
  \def\labelenumi{(\arabic{enumi})}
  \setcounter{enumi}{2}
  \tightlist
  \item
    All your poems are on the subject of soap bubbles. (premiss)
  \end{enumerate}
\item
  \begin{enumerate}
  \def\labelenumi{(\arabic{enumi})}
  \setcounter{enumi}{3}
  \tightlist
  \item
    No affected poetry is popular among people of real taste. (premiss)
  \end{enumerate}
\item
  \begin{enumerate}
  \def\labelenumi{(\arabic{enumi})}
  \setcounter{enumi}{4}
  \tightlist
  \item
    Only a modern poem would be on the subject of soap bubbles.
    (premiss)
  \end{enumerate}
\item
  \begin{enumerate}
  \def\labelenumi{(\arabic{enumi})}
  \setcounter{enumi}{5}
  \tightlist
  \item
    All your poems are modern poems. (from 3, 5)
  \end{enumerate}
\item
  \begin{enumerate}
  \def\labelenumi{(\arabic{enumi})}
  \setcounter{enumi}{6}
  \tightlist
  \item
    All your poems are affected. (from 2, 6)
  \end{enumerate}
\item
  \begin{enumerate}
  \def\labelenumi{(\arabic{enumi})}
  \setcounter{enumi}{7}
  \tightlist
  \item
    None of your poems are popular among people of real taste. (from 7,
    4)
  \end{enumerate}
\item
  \begin{enumerate}
  \def\labelenumi{(\arabic{enumi})}
  \setcounter{enumi}{8}
  \tightlist
  \item
    All interesting poems are popular among people of real taste. (from
    1)
  \end{enumerate}
\item
  \begin{enumerate}
  \def\labelenumi{(\arabic{enumi})}
  \setcounter{enumi}{9}
  \tightlist
  \item
    None of your poems are interesting. (from 8, 9)
  \end{enumerate}
\end{itemize}

\section{Validity and Soundness}\label{validity-and-soundness}

Covered So Far:

\begin{itemize}
\tightlist
\item
  Inference steps being deductively valid.
\item
  Therefore some premises deductively entail a conclusion.
\item
  This chapter informally explores validity and entailment
\end{itemize}

\subsection{Defining Validity}\label{defining-validity}

\begin{quote}
An inference step is \emph{valid} if and only if (iff) there is no
possible situation in which its premises would be true and its
conclusion false. Equivalently, in such a case, we will say that the
inference's premises \emph{entail} its conclusion.
\end{quote}

\section{Consistency and Equivalency}\label{consistency-and-equivalency}

\section{What is a proposition?}\label{what-is-a-proposition}

\begin{quote}
One or more propositions are (jointly) \emph{inconsistent} if and only
if there is no possible situation in which these propositions are all
true together.
\end{quote}

\begin{center}\rule{0.5\linewidth}{0.5pt}\end{center}

So what is a proposition?

Ayer, A. J. Language, Truth and Logic. Vol. 47. V. Gollancz, 1936.

\begin{quote}
For, if I am right, it will also follow that any sentence, whether of
the English or any other language, that is equivalent to \emph{s} can be
validly derived, in the language in question, from any sentence that is
equivalent to \emph{r}; and it this that my use of the word
``proposition'' indicates. (Page 7)
\end{quote}

\begin{itemize}
\tightlist
\item
  First a sentence
\item
  Next a statement (imperative, declarative, interrogative, exclamatory)
\item
  Empirically Verifiable
\item
  Therefore is always either true or false
\end{itemize}

Notice which ones can be true or false? Imperative: Go to your room!,
Interrogative: Did you have dinner with the victim the night of their
murder? Exclamatory: Oh darn! or Declarative?

\begin{center}\rule{0.5\linewidth}{0.5pt}\end{center}

Some Examples

If some propositions are consistent with each other, then adding a
further true proposition to them can't make them inconsistent.

\begin{itemize}
\tightlist
\item
  Yes?
\item
  No?
\end{itemize}

\begin{center}\rule{0.5\linewidth}{0.5pt}\end{center}

Consider:

\begin{itemize}
\tightlist
\item
  Socrates is a woman.
\item
  No women are philosophers.
\item
  Now consider:

  \begin{itemize}
  \tightlist
  \item
    Socrates is a philosopher
  \end{itemize}
\end{itemize}

Can both be true?

Either inconsistent, or one is false.

\subsection{Validity, Invalidity, and
Truth}\label{validity-invalidity-and-truth}

\textbf{\emph{The Invalidity Principle}}

\begin{quote}
The only combination ruled out by the definition of validity is a valid
inference step's having all true premisses and yet a false conclusion.
Deductive validity is about the necessary preservation of truth -- and
therefore a valid inference step cannot take us from actually true
premisses to an actually false conclusion. (\textbf{smit94?})
\end{quote}

\begin{center}\rule{0.5\linewidth}{0.5pt}\end{center}

\begin{itemize}
\item
  Either premises can be jointly consistent, if there is at least one
  situation where they can all be true together
\item
  There are inconsistent if there is no possible situation in which all
  are true together in that situation.
\end{itemize}

\textbf{\emph{Or premises taken with with their conclusion:}}

\begin{itemize}
\tightlist
\item
  Can the premises of an argument be true while at the same time,
  denying the truth of the conclusion?

  \begin{itemize}
  \tightlist
  \item
    Yes: Inconsistent
  \item
    No: Consistent
  \end{itemize}
\end{itemize}

\begin{center}\rule{0.5\linewidth}{0.5pt}\end{center}

\subsection{Equivalency}\label{equivalency}

\begin{quote}
Two propositions are equivalent iff they are true in exactly the same
possible situations.
\end{quote}

\begin{center}\rule{0.5\linewidth}{0.5pt}\end{center}

\subsection{Some Examples}\label{some-examples-1}

\begin{enumerate}
\def\labelenumi{(\arabic{enumi})}
\tightlist
\item
  If A entails C, and C is equivalent to C 0 , then A entails C 0 .
\item
  If A entails C, and A is equivalent to A0 , then A0 entails C.
\item
  If A and B entail C, and A is equivalent to A0 , then A0 and B entail
  C.
\end{enumerate}

\chapter{Ch. 3, Forms of Inference}\label{ch.-3-forms-of-inference}

\begin{enumerate}
\def\labelenumi{\arabic{enumi}.}
\tightlist
\item
  Whatever Donald Trump says is true.
\item
  Donald Trump says that the Flying Spaghetti Monster exists.
\end{enumerate}

So,

\begin{enumerate}
\def\labelenumi{\arabic{enumi}.}
\setcounter{enumi}{2}
\tightlist
\item
  It is true that the Flying Spaghetti Monster exists.
\end{enumerate}

\begin{center}\rule{0.5\linewidth}{0.5pt}\end{center}

Note three easy consequences of our definition of soundness:

\begin{enumerate}
\def\labelenumi{\arabic{enumi}.}
\tightlist
\item
  any sound argument has a true conclusion;
\item
  no pair of sound arguments can have conclusions inconsistent with each
  other;
\item
  no sound argument has inconsistent premisses.
\end{enumerate}

Why do these claims hold? For the following reasons:

\begin{enumerate}
\def\labelenumi{\arabic{enumi}.}
\setcounter{enumi}{3}
\tightlist
\item
  A sound argument starts from true premisses and involves a necessarily
  truth-preserving inference move -- so it must end up with a true
  conclusion.
\item
  Since a pair of sound arguments will have a pair of true conclusions,
  this means that the conclusions are true together. If they actually
  are true together, then of course they can be true together. And if
  they can be true together then (by definition) the conclusions are
  consistent with each other.
\item
  Since inconsistent premisses cannot all be true together, an argument
  starting from those premisses cannot satisfy the first of the
  conditions for being sound.
\end{enumerate}

\section{More Forms of Inference
(Examples)}\label{more-forms-of-inference-examples}

\chapter{Exams}\label{exams}

\paragraph{Chapter 1}\label{chapter-1}

What are the premisses, inference markers, and conclusions of the
following arguments? Which of these arguments do you suppose involve
deductively valid reasoning? Why?

\begin{quote}
Some forms or patterns of inference are deductively reliable, then,
meaning that every inference step which is an instance of the same
pattern is valid (Smith 2021, 6).
\end{quote}

(Just improvise, and answer the best you can!)

\begin{enumerate}
\def\labelenumi{(\arabic{enumi})}
\item
  The Democrats will win the election. There's only a week to go. The
  polls put them 20 points ahead, and a lead of 20 points with only a
  week to go to polling day can't be overturned.
\item
  Most pelicans are corrupt. After all, most ordinary birds are corrupt
  -- and pelicans are ordinary birds.
\item
  Anyone who is well prepared for the race, even if she doesn't come in
  first, will at least come in the top 10. Jane is well prepared, so she
  will at least come in the top ten.
\end{enumerate}

\paragraph{Chapter 2}\label{chapter-2}

Which of the following claims are true and which are false? Explain why
the true claims hold good, and give counterexamples to the false claims.

\begin{enumerate}
\def\labelenumi{(\arabic{enumi})}
\setcounter{enumi}{3}
\item
  If an argument has false premisses and a true conclusion, then the
  truth of the conclusion can't really be owed to the premisses: so the
  argument cannot really be valid.
\item
  You can make an invalid argument valid by adding extra premisses.
\end{enumerate}

\paragraph{Chapter 3}\label{chapter-3}

Which of the following patterns of inference are deductively reliable,
meaning that all their instances are valid? (Here `F', `G' and `H' hold
the places for general terms.) If you suspect an inference pattern is
unreliable, find an instance which has to be invalid because it has true
premises and a false conclusion.

\begin{enumerate}
\def\labelenumi{(\arabic{enumi})}
\setcounter{enumi}{5}
\tightlist
\item
  Some F are G; no G is H; so, some F are not H.
\end{enumerate}

\paragraph{Chapter 4}\label{chapter-4}

Which of the following arguments are valid? Where an argument is valid,
sketch an informal proof. Some of the examples are enthymemes that need
repair.

\begin{enumerate}
\def\labelenumi{(\arabic{enumi})}
\setcounter{enumi}{6}
\item
  Only logicians are good philosophers. No existentialists are
  logicians. Some existentialists are French philosophers. So, some
  French philosophers are not good philosophers.
\item
  No philosopher is illogical. Jones keeps making argumentative
  mistakes. No logical person keeps making argumentative mistakes. All
  existentialists are philosophers. So, Jones is not an existentialist.
\end{enumerate}

\paragraph{Chapter 6}\label{chapter-6}

\begin{enumerate}
\def\labelenumi{(\arabic{enumi})}
\setcounter{enumi}{8}
\tightlist
\item
  Only logicians are wise. Some philosophers are not logicians. All who
  love Aristotle are wise. Hence some of those who don't love Aristotle
  are still philosophers.
\end{enumerate}

\paragraph{Chapter 8}\label{chapter-8}

Give negations of the following in natural English:

\begin{enumerate}
\def\labelenumi{(\arabic{enumi})}
\setcounter{enumi}{9}
\tightlist
\item
  It is not the case that both Jack and Jill went up the hill.
\end{enumerate}

Using the provided interpretations, render the following into formalized
language:

P: Peter loves Jane.\\
Q: Jane loves Peter.\\
R: Jesse loves Jane.\\
S: Peter is wise.

\begin{enumerate}
\def\labelenumi{(\arabic{enumi})}
\tightlist
\item
  Peter doesn't love Jane.
\item
  Peter is wise and he loves Jane.
\item
  Either Peter loves Jane or Jesse does.
\item
  Peter and Jane love each other.
\item
  Neither Peter loves Jane nor does Jesse.
\item
  It isn't the case that Peter loves Jane nor does Jane love Peter.
\item
  Either Peter is not wise or both he and Jesse love Jane.
\item
  It isn't the case that either Peter loves Jane or Jane loves Peter.
\end{enumerate}

\part{Introduction to Philosophy}

\chapter{Introduction}\label{introduction-2}

\section{The Cave}\label{the-cave}

\textbf{\emph{Descending back into Plato's Cave:}} Onyx's glowing secret
cave (\href{https://youtu.be/TKWbqwo1kQY?si=4oNUjX7TRaC9VTPi}{silent
looping ambient video from \emph{Star Citizen}})

\section{}\label{section}

🎯 Learning Objectives

\begin{itemize}
\tightlist
\item
  Develop critical thinking and reasoning skills
\item
  Improve reading, writing, and discussion abilities
\item
  Recognize and apply philosophical methods
\end{itemize}

\subsection{No Lectures}\label{no-lectures}

You are about to enter The Cave of Wonder and Reason. Forget who you
where, here --- every cave, dungeon, dream, room, forces you to answer a
question philosophers have bled over for 2,500 years. \emph{What is
consciousness}, \emph{What is reality?}, \emph{What is knowledge?}.

\subsection{}\label{section-1}

Here's the deal:

You roll dice. You defend a position. Is it yours? Who cares. You make
choices, are they yours? No, they belong to your avatar. You question
your beliefs? No, you question your avatar's beliefs. You eat your
friend.

This is not a game. It's a philosophy lab inside an experience machine.

By December, you'll have lived Socrates, Descartes, Mill, and Rawls ---
not just read them.

Pick one, this is your character for the rest of the course. That
character's beliefs are your beliefs. In fact, its better that we don't
share our personal beliefs. Make up some. Write your name on them.

Today, we begin in Room A. A voice echoes:

\begin{itemize}
\tightlist
\item
  ``What is philosophy?''
\end{itemize}

Roll a d20. Tell me \textbf{\emph{an}} answer.

\chapter{Syllabus}\label{syllabus-1}

\section{Welcome to Introduction to
Philosophy}\label{welcome-to-introduction-to-philosophy}

In this course, we will use David Chalmer's book
\textbf{\emph{Reality+}} to look at questions humans have been asking
for thousands of years. While the primary focus of the book is look at
one question in particular ``Are we in a simulation?'', we use this
question to frame others such as: ``What does it mean to know that the
sky is blue?'', ``is murder really wrong?'', ``How did I come to know
that murder is wrong?''.

\section{💻 Class Meeting Times}\label{class-meeting-times}

\begin{itemize}
\tightlist
\item
  📆 Mondays and Wednesdays
\item
  ⌚ 10:30am - 11:45am
\end{itemize}

\section{Course Convenor}\label{course-convenor}

Dr.~Monty Reynolds 📧 mreynolds1@stetson.edu

Office Hours:

\begin{itemize}
\tightlist
\item
  When:

  \begin{itemize}
  \tightlist
  \item
    Tuesday: 1:00--3:00 PM
  \item
    Thursday: 1:00--3:00 PM
  \end{itemize}
\item
  Where: Elizabeth Hall 104
\item
  How to book: Drop in, email, or book via
  \href{https://outlook.office.com/bookwithme/user/59570715559b43fdabc651a89a3ed839@stetson.edu?anonymous&ismsaljsauthenabled&ep=plink}{Microsoft
  Bookings}
\end{itemize}

\section{Course Information}\label{course-information}

Introduction to Philosophy: Value, Meaning, and Humanity's Place in the
Modern World

Times: 10:30-11:45 Days: Mons and Weds or Tues Thurs Where: Davis Hall
209

\section{Required Texts:}\label{required-texts}

Chalmers,~David J..~Reality+: Virtual Worlds and the Problems of
Philosophy.~United Kingdom:~Penguin Books Limited,~2022.

Here is an amazon link to the book:
https://www.amazon.com/Reality-Virtual-Worlds-Problems-Philosophy/dp/0393635805

This will also be supplemented with handouts.

\section{Expectations:}\label{expectations}

Come prepared to engage with assigned readings in class, referencing
specific passages as prompted by the instructor.

Bring physical or digital copies of readings to class for annotation and
short reflections.

Submit all assignments via Canvas by the due date.

Active participation and regular attendance are essential for success.

Success: Active participation, timely submissions, and attendance are
key.

\section{Grading Calculation}\label{grading-calculation}

\chapter{Weeklies}\label{weeklies}

\textbf{\emph{Weekly Reflections (32 points):}} These will be done each
week in class. Each reflection is worth a possible total of 8 points.
There are 12 possible reflections. I will only grade 8 or your best
reflections for a possible total of 32 points.

\textbf{\emph{Essay 1 \& 2 (80 points):}} Each Quiz is worth a possible
total of 10 points per category, times 4 categories equals 40 points
times 2 Quizzes.

\begin{itemize}
\item
  Your paper should consist of two distinct parts. The first part, which
  should account for approximately half the length of the paper, should
  be a short explication of the essay you have chosen. This part should
  contain a statement of the author's position, a statement of your own
  position (tell me whether you agree or disagree with the author) and
  an explanation of the author's supporting arguments.
\item
  The second half of the paper should consist of your own evaluation or
  critique of the essay. In this part of the paper you should tell me
  WHY you agree or disagree with the author. If you agree with the
  author you should tell me what argument(s) are convincing. Then tell
  me what argument(s) might be proffered by one who disagrees with the
  author and how the author might respond.
\item
  If you disagree with the author, you should provide criticism of the
  author's essay and attempt to explain how the author might respond to
  the criticism. Hence, whether you agree or disagree with the author,
  you should provide arguments against the author and responses thereto.
  Finally, you should explain why you believe your position is the best
  position. You must NOT use outside sources for this essay. I will
  provide a more detailed account in class.
\end{itemize}

\href{./critical-reflections/critical-reflection-1.qmd}{See here}

\chapter{Quizzes}\label{quizzes}

\textbf{\emph{Quiz 1 \& 2 (80 points):}} Each Quiz is worth a possible
total of 10 points per category, times 4 categories equals 40 points
times 2 Quizzes.

\textbf{\emph{Presentation:}} Your grade in this area depends on your
preparation and participation in the DND class meetings.
\href{https://drive.google.com/drive/folders/1ZAXxdpvv5jzylj8_18jXUTI68o9MLrXw?usp=sharing}{We
will use the dnd adventure sheet linked here to gage participation}.



Students will be evaluated based on a total of 640 points, with the
final grade determined by the percentage of points earned. The
components are as follows:

\subsection{Grading Rubric}\label{grading-rubric}

\begin{longtable}[]{@{}
  >{\raggedright\arraybackslash}p{(\columnwidth - 8\tabcolsep) * \real{0.2000}}
  >{\raggedright\arraybackslash}p{(\columnwidth - 8\tabcolsep) * \real{0.2000}}
  >{\raggedright\arraybackslash}p{(\columnwidth - 8\tabcolsep) * \real{0.2000}}
  >{\raggedright\arraybackslash}p{(\columnwidth - 8\tabcolsep) * \real{0.2000}}
  >{\raggedright\arraybackslash}p{(\columnwidth - 8\tabcolsep) * \real{0.2000}}@{}}
\toprule\noalign{}
\begin{minipage}[b]{\linewidth}\raggedright
\end{minipage} & \begin{minipage}[b]{\linewidth}\raggedright
Excellent 4
\end{minipage} & \begin{minipage}[b]{\linewidth}\raggedright
Good 3
\end{minipage} & \begin{minipage}[b]{\linewidth}\raggedright
Needs Improvement 2
\end{minipage} & \begin{minipage}[b]{\linewidth}\raggedright
Unacceptable 1
\end{minipage} \\
\midrule\noalign{}
\endhead
\bottomrule\noalign{}
\endlastfoot
CONTENT and Argument & & & & \\
Thesis & A clear statement of the main conclusion of the paper. & The
thesis is obvious, but there is no single clear statement of it. & The
thesis is present, but must be uncovered or reconstructed from the text
of the paper. & There is no thesis. \\
Premises & Each reason for believing the thesis is made clear, and as
much as possible, presented in single statements. It is also clear which
premises are to be taken as given, and which will be supported by
sub-arguments. The paper provides sub-arguments for controversial
premises. If there are sub-arguments, the premises for these are clear,
and made in single statements. The premises which are taken as given are
at least plausibly true. & The premises are all clear, although each may
not be presented in a single statement. It is also pretty clear which
premises are to be taken as given, and which will be supported by
sub-arguments. The paper provides sub-arguments for controversial
premises. If there are sub-arguments, the premises for these are clear.
The premises which are taken as given are at least plausibly true. & The
premises must be reconstructed from the text of the paper. It is not
made clear which premises are to be taken as given, and which will be
supported by sub-arguments. There are no sub-arguments, or, if there are
sub-arguments, the premises for these are not made clear. The paper does
not provide sub-arguments for controversial premises. The plausibility
of the premises which are taken as given is questionable. & There are no
premises---the paper merely restates the thesis. Or, if there are
premises, they are much more likely to be false than true. \\
Support & The premises clearly support the thesis, and the author is
aware of exactly the kind of support they provide. The argument is
either valid as it stands, or, if invalid, the thesis, based on the
premises, is likely to be or plausibly true. & The premises support the
thesis, and the author is aware of the general kind of support they
provide. The argument is either valid as it stands, or, if invalid, the
thesis, based on the premises, is likely to be or plausibly true. & The
premises somewhat support the thesis, but the author is not aware of the
kind of support they provide. The argument is invalid, and the thesis,
based on the premises, is not likely to be or plausibly true. & The
premises do not support the thesis. \\
Counter-Arguments & The paper considers both obvious and unobvious
counter-examples, counter-arguments, and/or opposing positions, and
provides original and/or thoughtful responses. & The paper considers
obvious counter-examples, counter-arguments, and/or opposing positions,
and provides responses. & The paper may consider some obvious
counter-examples, counter-arguments, and/or opposing positions, but some
obvious ones are missed. Responses are non-existent or mere claims of
refutation. & No counter-examples, counter-arguments, or opposing
positions are considered. \\
\end{longtable}

\paragraph{Attendance (5\% of final grade, 32
points):}\label{attendance-5-of-final-grade-32-points}

\begin{itemize}
\tightlist
\item
  Based on unexcused absences throughout the semester.
\end{itemize}

Grading Scale (equal increments of 8 points):

\begin{longtable}[]{@{}lll@{}}
\toprule\noalign{}
\endhead
\bottomrule\noalign{}
\endlastfoot
0--1 & unexcused absences: & 32 points \\
2 & unexcused absences: & 24 points \\
3--4 & unexcused absences: & 16 points \\
5--6 & unexcused absences: & 8 points \\
6 & unexcused absences: & 0 points \\
\end{longtable}

\subsection{Reading Schedule}\label{reading-schedule}

\textbf{\emph{Adjust all dates by one for Tuesday Thursday Course}}

\textbf{\emph{Weeklies Due Every Monday Starting Jan 19}}

Wed Jan 14, handouts

Mon Jan 19 MLK Day, no classes

Wed Jan 21, Chapter 2, Simulation Hypothesis

Mon Jan 26, Chapter 3, Knowledge

Wed Jan 28, The external world, ch.~4

Mon Feb 2, ch.~5, Possible realities, bostrom and Moravec

Wed Feb 4, ch.~6, What is Reality?

Mon Fed 9, ch.~7, Is God a hacker in the universe up?

Wed Fed 11, ch.~8 Information, 2nd DND Adventure, Student Led

Mon Feb 16, ch.~9, On Bits

Wed Feb 18, ch.~10, Reality and Virtual Reality

Mon Feb 23, ch.~11, Illusion Machines

Wed Feb 25, Quiz 1

Mon Mar 02, Spring Break

Wed Mar 4, Spring Break

Mon Mar 9, ch.~13 ch.~12, 3rd DND Adventure, Student Led

Wed Mar 11, ch.~14, Mind and body Realism inside a virtual universe

Mon Mar 16, ch.~15, 4th DND Adventure, Student Led

Wed Mar 18, ch.~16, The extended mind hypothsis

Mon Mar 23, ch.~17 Critical Reflection Due

Wed Mar 25, ch.~18, Virtual ethics and intentiality

Mon Mar 30, ch.~19, Social Ontology inside virtual worlds

Wed Apr 1, ch.~20, 5th DND Adventure on Sense and Reference, Student Led

Mon Apr 6, ch.~21, Cause and effect inside virtual systems

Wed Apr 8, ch.~22, Mathematical, physical and cultural structualism

Mon Apr 13, ch.~23, Virtual Eden?

Wed Apr 15, ch.~24, Brains in a Vat

Mon Apr 20, TBD

Wed Apr 22, TBD

Mon Apr 27, TBD

Wed Apr 29, last day of classes, Final Quiz

Paper Due on Day of Final Quiz (Submit on Canvas)

\subsection{University Supports and
Policies}\label{university-supports-and-policies}

Please Note: Oral appeals for grade changes are NOT accepted. If you
feel that you have an extremely strong case for a grade change, you may
submit a one-page essay explaining your reasons to me within one week of
your work being returned.

A.Honor Code: University rules concerning scholastic dishonesty will be
strictly followed. Any attempt at academic cheating will be referred to
the Honor Council. I expect you to abide by the Honor Code that we have
adopted here at Stetson. Please sign all work ``Pledged, your name.''
University rules concerning scholastic dishonesty will be strictly
followed. As a community of intellectual inquirers, we are all committed
to academic integrity and honesty. The Stetson University Student
Government Association, on its own initiative, recently drafted a
proposal for an honor code, a powerful statement of the student body's
commitment to this community of honor and integrity. For this reason,
all work that you turn in must be your own. Any contribution from others
must be clearly acknowledged. Unauthorized assistance on exams
(take-home as well as in-class), research papers, or projects can be
neither given nor received. The University has fully endorsed its honor
system and abides by its precepts. If you have questions regarding the
method/procedure for citing the work of others, please feel free to
consult us.

The Department of Philosophy expects academic honesty of all students;
we expect all students to abide by Stetson's honor code and to adhere to
the honor system. Academic dishonesty includes, but is not limited to,
altering or misusing documents, plagiarizing, misrepresentation,
knowingly providing false information, colluding, and other forms of
cheating. Students should read and follow Stetson's definitions and
policies on `Academic Honesty,' the honor code, and the honor system in
Connections: The Campus Life Handbook and Calendar. It is the student's
responsibility to be familiar with these definitions and policies;
ignorance of them is not an acceptable excuse for violating them. Cases
of suspected academic dishonesty, including plagiarism, will be referred
to the Academic Honor Council, as per departmental policy.

The honor code can be found online at the following
address:http://www.stetson.edu/honorsystem/
http://www.stetson.edu/honorsystem/

B.Assignment standards: Late Work: All assignments must be handed in on
time: late work will be docked a half-letter grade per day unless you
get my approval for an extension before the due date. (Student athletes
and others on school business: please, be advised to turn your
assignments in BEFORE you leave, not after you return). No make-up exams
or quizzes will be given without documented University accepted excuse
(and they will differ from the ones given in class). This includes group
quizzes.

C.Accommodation: Writing Center: Many students experience difficulty
writing research papers. For that reason, Stetson University supports a
Writing Center, located on the Second Floor of the Library. I have found
that one or two visits to the Writing Center can make a tremendous
difference in the quality of a paper. If you find that you have trouble
with your written assignments, I strongly encourage you to seek the help
of the Writing Center. The Academic Success Center provides academic and
disability resources for all Stetson University Students. Students who
anticipate barriers related to the format or requirements of a course
should meet with me to discuss ways to ensure full participation. If
disability-related accommodations are necessary, please register with
the Academic Success Center (386-822-7127; www.stetson.edu/asc). You and
I, and the ASC, will plan how best to coordinate accommodations.
Sheridan's writing center hours: W 2-6, F 12-3.

The ASC also coordinates free tutoring on campus for students. You can
meet with a tutor to review principles, learn content-specific study
strategies, and enhance content area knowledge. To review the tutoring
options available and schedules, please see our website
www.stetson.edu/asc/tutoring/php.

2.Counseling: Counseling: College can be extremely stressful for
students, especially if this is the first time you've been away from
home for an extended period of time or if there are other pressures that
you are facing. For this reason, you may find it helpful to consult the
University Counseling Center. Here is their contact information:

Phone number: 386-822-8900

Location: The office is located in Griffith Hall

Office hours: Weekdays from 8:00 a.m. to 4:30 p.m.

If a student experiences a mental health emergency after hours, they can
simply call Public Safety (386-822-7300) and ask to speak with the
on-call counselor.

We are staffed with qualified professional counselors who are trained to
support and guide students through difficult transitions, experiences,
and feelings.

Counseling is confidential and free of charge for all currently enrolled
Stetson University students.

If experiencing a mental health crisis, you may also call 386-822-8740
to be connected to Volusia County Crisis Response Team: If having
thoughts of harming yourself or others, select option one. For all other
mental health needs select option nine.

For medical emergencies call 911.

Schedule of Readings/Assignments: subject to change as needed, it is
YOUR responsibility to regularly check blackboard for announcements.

Students are required to check their Stetson email and Canvas accounts
frequently, daily if possible. Do not use technology as an excuse for
not completing an assignment/quiz or for submitting an assignment/quiz
late or improperly. Late assignments/submissions are not accepted.

All students will be asked to verify in an online quiz that they agree
and will abide by the procedures and policies presented in this
syllabus. If you do not feel you can abide by the course rules, please
drop the course.

Any recordings of this class (audio, video, or otherwise) may ONLY be
used for personal academic use. Recordings may NOT be shared with other
people without written consent from the professor. The information
contained in recordings constitutes intellectual property and is
protected under federal copyright laws. This information may NOT be
published or quoted without the express written consent of the professor
and without giving proper identification and credit to the professor.
Recordings of this class may not be used in any way against the faculty
member, teaching assistant, other lecturer, or students whose classroom
comments are recorded as part of the class. Violation of these
stipulations will result in being reported to the Office of Community
Standards.

D. Food Insecurity: At Stetson University, we are committed to
supporting students' well-being, including ensuring access to nutritious
food. If you are experiencing a lack of access to safe and nutritious
food, the Hatters Helping Hatters Food Pantry is her to help. The Food
Pantry is a free, no-questions-asked supplemental food resource for all
Stetson students, staff \& faculty. Food Pantry location:

\begin{itemize}
\tightlist
\item
  Carlton Union Building (CUB 278), open 24 hours.
\end{itemize}

\chapter{Lectures}\label{lectures-1}

\chapter{Introductions}\label{introductions}

\chapter{Games That Go Too Far}\label{games-that-go-too-far}

\chapter{What is the simulation
hypothesis}\label{what-is-the-simulation-hypothesis}

\section{Possible Worlds and Thought
Experiments}\label{possible-worlds-and-thought-experiments}

\section{Simulations in Science
Fiction}\label{simulations-in-science-fiction}

\section{The Simulation Hypothesis}\label{the-simulation-hypothesis}

\section{Can you prove a negative?}\label{can-you-prove-a-negative}

\section{Can you prove a positive?}\label{can-you-prove-a-positive}

\section{Simulation Hypothesis, a scientific
hypothesis}\label{simulation-hypothesis-a-scientific-hypothesis}

\section{Simulation and virtual
worlds}\label{simulation-and-virtual-worlds}

\chapter{The Problem of the External
World}\label{the-problem-of-the-external-world}

\section{\texorpdfstring{The \emph{(very silly)} Evolutionary Argument
from Naturalism
(EAAN)}{The (very silly) Evolutionary Argument from Naturalism (EAAN)}}\label{the-very-silly-evolutionary-argument-from-naturalism-eaan}

\begin{itemize}
\tightlist
\item
  Philosophical Naturalism (N):

  \begin{itemize}
  \tightlist
  \item
    There are no supernatural beings (i.e., God, angeles, demons,
    ghosts)
  \end{itemize}
\item
  Contemporary Evolutionary Theory (E)

  \begin{itemize}
  \tightlist
  \item
    Human beings have evolved in conformity with current evolutionary
    doctrine
  \end{itemize}
\item
  Naturalism and Evolutionary theory are at odds with one another.

  \begin{itemize}
  \tightlist
  \item
    I.e., if philosophical naturalism is true, then contemporary
    evolutionary theory can't be true.
  \item
    If Contemporary evolutionary theory is true, then naturalism cannot
    be true.
  \end{itemize}
\end{itemize}

Therefore, either E is true, or N is true, but not both.

\section{Reliability of Cognitive
Faculties}\label{reliability-of-cognitive-faculties}

\begin{itemize}
\tightlist
\item
  Memory
\item
  Perception
\item
  Reason
\end{itemize}

We often ground a good majority of our beliefs in these faculties and
more.

As such, these faculties are reliable if the majority of our beliefs
that are grounded in them are true.

\begin{center}\rule{0.5\linewidth}{0.5pt}\end{center}

However,

according to E:

\begin{itemize}
\tightlist
\item
  human life developed from aboriginal unicellular life

  \begin{itemize}
  \tightlist
  \item
    natural selection
  \item
    genetic drift
  \item
    genetic variation (random genetic mutatation)
  \end{itemize}
\item
  Natural selection discards most mutations

  \begin{itemize}
  \tightlist
  \item
    Some mutations have adaptive value
  \item
    some do not
  \end{itemize}
\item
  Through such processes, most if not all organic life as developed
\end{itemize}

\begin{center}\rule{0.5\linewidth}{0.5pt}\end{center}

\begin{itemize}
\tightlist
\item
  Through such processes, our cognitive faculties too, have evolved
\item
  But natural selection is not interested in truth but rather adaptive
  behavior, the four Fs:

  \begin{itemize}
  \tightlist
  \item
    Feeding,
  \item
    Fleeing,
  \item
    Fighting
  \item
    Reproducing
  \end{itemize}
\item
  It is possible that some other function has proved more adaptive
\item
  Therefore we cannot be sure we can trust our faculties to be reliable
\end{itemize}

\begin{center}\rule{0.5\linewidth}{0.5pt}\end{center}

For instance, here is Charles Darwin on the matter:

\begin{quote}
the horrid doubt always arises whether the convictions of man's mind,
which has been developed from the mind of the lower animals, are of any
value or at all trutworthy. Would any one trust in the convictions of a
monkey's mind, if there are any convictions in such a mind? {[}Letter to
William Graham, Down, July 3, 1881, in The Life and Letters of Charles
Darwin INcluding an Autobiographical Chapter, ed.~Francis Darwin
(London: John Murray, Albermarle Street, 1887), I:315-316.{]}
\end{quote}

\begin{center}\rule{0.5\linewidth}{0.5pt}\end{center}

Here is Patricia Churchland:

\begin{quote}
Boiled down to essentials, a nervous system enables the organism to
succeed in the four F's: fedding, fleeing, fighting and reproducing. The
principle chore of nervous systems is the get the body parts where they
should be in order that the organism may survive . . . Improvements in
sensorimotor control confer an evolutionary advantage: a fancier style
of representing is advantageous so long as it is geared to the
organism's way of life and enhances the organism's chances of survival.
Truth, whatever that is, definitely takes the hindmost. {[}Churchland,
``Epistemology in the Age of Neuroscience,'' Journal of Philosophy 84
(October 1987): 548{]}
\end{quote}

\section{Descartes}\label{descartes}

\begin{quote}
Certainly, up to now whatever I have accepted as fully true I have
learned either from or by means of the senses: but I have discovered
that they sometimes deceive us, and prudence dictataes that we should
never fully trust those who have deceived us even once.
\end{quote}

\section{Berkeley}\label{berkeley}

\section{G.E. Moore}\label{g.e.-moore}

\section{Alvin Plantinga}\label{alvin-plantinga}

\chapter{The Problem of the External
World}\label{the-problem-of-the-external-world-1}

\section{\texorpdfstring{The \emph{(very silly)} Evolutionary Argument
from Naturalism
(EAAN)}{The (very silly) Evolutionary Argument from Naturalism (EAAN)}}\label{the-very-silly-evolutionary-argument-from-naturalism-eaan-1}

\begin{itemize}
\tightlist
\item
  Philosophical Naturalism (N):

  \begin{itemize}
  \tightlist
  \item
    There are no supernatural beings (i.e., God, angeles, demons,
    ghosts)
  \end{itemize}
\item
  Contemporary Evolutionary Theory (E)

  \begin{itemize}
  \tightlist
  \item
    Human beings have evolved in conformity with current evolutionary
    doctrine
  \end{itemize}
\item
  Naturalism and Evolutionary theory are at odds with one another.

  \begin{itemize}
  \tightlist
  \item
    I.e., if philosophical naturalism is true, then contemporary
    evolutionary theory can't be true.
  \item
    If Contemporary evolutionary theory is true, then naturalism cannot
    be true.
  \end{itemize}
\end{itemize}

Therefore, either E is true, or N is true, but not both.

\section{Reliability of Cognitive
Faculties}\label{reliability-of-cognitive-faculties-1}

\begin{itemize}
\tightlist
\item
  Memory
\item
  Perception
\item
  Reason
\end{itemize}

We often ground a good majority of our beliefs in these faculties and
more.

As such, these faculties are reliable if the majority of our beliefs
that are grounded in them are true.

\begin{center}\rule{0.5\linewidth}{0.5pt}\end{center}

However,

according to E:

\begin{itemize}
\tightlist
\item
  human life developed from aboriginal unicellular life

  \begin{itemize}
  \tightlist
  \item
    natural selection
  \item
    genetic drift
  \item
    genetic variation (random genetic mutatation)
  \end{itemize}
\item
  Natural selection discards most mutations

  \begin{itemize}
  \tightlist
  \item
    Some mutations have adaptive value
  \item
    some do not
  \end{itemize}
\item
  Through such processes, most if not all organic life as developed
\end{itemize}

\begin{center}\rule{0.5\linewidth}{0.5pt}\end{center}

\begin{itemize}
\tightlist
\item
  Through such processes, our cognitive faculties too, have evolved
\item
  But natural selection is not interested in truth but rather adaptive
  behavior, the four Fs:

  \begin{itemize}
  \tightlist
  \item
    Feeding,
  \item
    Fleeing,
  \item
    Fighting
  \item
    Reproducing
  \end{itemize}
\item
  It is possible that some other function has proved more adaptive
\item
  Therefore we cannot be sure we can trust our faculties to be reliable
\end{itemize}

\begin{center}\rule{0.5\linewidth}{0.5pt}\end{center}

For instance, here is Charles Darwin on the matter:

\begin{quote}
the horrid doubt always arises whether the convictions of man's mind,
which has been developed from the mind of the lower animals, are of any
value or at all trutworthy. Would any one trust in the convictions of a
monkey's mind, if there are any convictions in such a mind? {[}Letter to
William Graham, Down, July 3, 1881, in The Life and Letters of Charles
Darwin INcluding an Autobiographical Chapter, ed.~Francis Darwin
(London: John Murray, Albermarle Street, 1887), I:315-316.{]}
\end{quote}

\begin{center}\rule{0.5\linewidth}{0.5pt}\end{center}

Here is Patricia Churchland:

\begin{quote}
Boiled down to essentials, a nervous system enables the organism to
succeed in the four F's: fedding, fleeing, fighting and reproducing. The
principle chore of nervous systems is the get the body parts where they
should be in order that the organism may survive . . . Improvements in
sensorimotor control confer an evolutionary advantage: a fancier style
of representing is advantageous so long as it is geared to the
organism's way of life and enhances the organism's chances of survival.
Truth, whatever that is, definitely takes the hindmost. {[}Churchland,
``Epistemology in the Age of Neuroscience,'' Journal of Philosophy 84
(October 1987): 548{]}
\end{quote}

\section{Descartes}\label{descartes-1}

\begin{quote}
Certainly, up to now whatever I have accepted as fully true I have
learned either from or by means of the senses: but I have discovered
that they sometimes deceive us, and prudence dictataes that we should
never fully trust those who have deceived us even once.
\end{quote}

\section{Berkeley}\label{berkeley-1}

\section{G.E. Moore}\label{g.e.-moore-1}

\section{Alvin Plantinga}\label{alvin-plantinga-1}

\chapter{Sims}\label{sims}

\begin{itemize}
\tightlist
\item
  From the inside, \emph{SimUniverse} will be indistinguishable from the
  univere it is a simulation of.
\item
  Simulation of universe contains 10 billion people, one for each person
  in the actual universe.
\item
  Say that it is a very popular program, then millions if not billions
  may have \emph{SimUniverse} on their devices.
\item
  Therefore, there will be many more sims than non sims.
\end{itemize}

\chapter{What is happening in this
Chapter?!}\label{what-is-happening-in-this-chapter}

\section{Questions Philosophers Ask}\label{questions-philosophers-ask}

\begin{itemize}
\tightlist
\item
  What is Real?
\item
  How do we know it is real?
\item
  Why should we care?
\end{itemize}

\begin{center}\rule{0.5\linewidth}{0.5pt}\end{center}

\begin{itemize}
\tightlist
\item
  Metaphysics
\item
  Epistemology
\item
  Value Theory
\end{itemize}

\begin{center}\rule{0.5\linewidth}{0.5pt}\end{center}

But notice that it is not just about providing answers to questions.
Sometimes it is how we provide an answer that is even more interesting.

\begin{center}\rule{0.5\linewidth}{0.5pt}\end{center}

There are two very important terms here:

\begin{itemize}
\tightlist
\item
  If so, then we are \emph{probably sims}
\item
  It will \emph{never} happen
\end{itemize}

\begin{center}\rule{0.5\linewidth}{0.5pt}\end{center}

One is a special kind of possibility, while the other entails a
necessity.

\begin{center}\rule{0.5\linewidth}{0.5pt}\end{center}

The Argument:

\begin{enumerate}
\def\labelenumi{\arabic{enumi}.}
\tightlist
\item
  At least one in ten nonsim populations will each create a thousand sim
  populations.
\item
  If at least one in ten nonsim populations will each create a thousand
  sim populations, then at least 99 percent of intelligent beings are
  sims.
\item
  If at least 99 percent of intelligent beings are sims, we are
  \emph{probably} sims.
\item
  Therefore, we are \emph{probably} sims.
\end{enumerate}

\section{Nick Bostrom}\label{nick-bostrom}

At least one of the following propositions is true (a proposition is a
statement that is either true or false)

\begin{enumerate}
\def\labelenumi{\arabic{enumi}.}
\tightlist
\item
  The human species is very likely to go extinct before reaching a
  ``posthuman'' stage;
\item
  Any posthuman civilization is extremely unlikely to run a significant
  number of simulations of their evolutionary history (or variations
  thereof);
\item
  We are almost certainly living in a computer simulation.
\end{enumerate}

\begin{center}\rule{0.5\linewidth}{0.5pt}\end{center}

1 and 2 are sim blockers:

\begin{enumerate}
\def\labelenumi{\arabic{enumi}.}
\tightlist
\item
  Nonsims (humans) likely to go extinct before creating nonsims.
\item
  Nonsims likely to choose not to create sims.
\end{enumerate}

\chapter{Strong vs.~Weak
Verificationism}\label{strong-vs.-weak-verificationism}

By now, as before, you may be criticizing the following argument:

\begin{enumerate}
\def\labelenumi{\arabic{enumi}.}
\tightlist
\item
  We normally rely on our senses for evidence
\item
  But Bostrom's argument does not rely on his senses
\item
  How then does he arrive at his conclusion?
\end{enumerate}

\begin{center}\rule{0.5\linewidth}{0.5pt}\end{center}

And you are probably used to the following inference patters:

\begin{enumerate}
\def\labelenumi{\arabic{enumi}.}
\tightlist
\item
  Some experience, there is water falling on my face.
\item
  Some belief / conclusion, it is raining today.
\end{enumerate}

\begin{center}\rule{0.5\linewidth}{0.5pt}\end{center}

So your counter looks like this:

\begin{center}\rule{0.5\linewidth}{0.5pt}\end{center}

\begin{enumerate}
\def\labelenumi{\arabic{enumi}.}
\tightlist
\item
  1 by 1 nonsim sim populations?
\item
  SimUniverses populated by a intelligent sims?
\item
  It certainly looks like 1 and 2 are super empirical (beyond sense
  experience).
\end{enumerate}

\begin{center}\rule{0.5\linewidth}{0.5pt}\end{center}

Here, Ayer agrees with you:

\begin{center}\rule{0.5\linewidth}{0.5pt}\end{center}

But this also means:

\begin{enumerate}
\def\labelenumi{\arabic{enumi}.}
\tightlist
\item
  We need a new criterion of meaning
\item
  A formal syntax
\end{enumerate}

\begin{quote}
Philosophy of logic is devoted to the investigation, analysis and
reflection on issues arising in logic, while philosophical logic
concerns questions about reference, truth, quantification, existence,
entailment, predication, identity, modality, and necessity. A typical
example of philosophical logic is the application of formal logical
techniques to philosophical problems.
\href{https://philpapers.org/browse/logic-and-philosophy-of-logic}{PhilPapers
Logic Entry}
\end{quote}

\begin{center}\rule{0.5\linewidth}{0.5pt}\end{center}

Here are those words again:

\begin{quote}
But it is verifiable in the weak sense, if it is \emph{possible} for
experience to render it \emph{probable}.
\end{quote}

\chapter{Summary of Chapter 5}\label{summary-of-chapter-5}

\section{Simulation Argument}\label{simulation-argument}

\begin{enumerate}
\def\labelenumi{\arabic{enumi}.}
\tightlist
\item
  Simulation technology is likely to be so ubiquitous that most beings
  in the universe (or most beings with experiences like ours) are sims.
\item
  Therefore, we are probably sims.
\end{enumerate}

\section{Three Objections:}\label{three-objections}

\begin{itemize}
\tightlist
\item
  It will never happen
\item
  We're special, have special features that cannot be simulated
\item
  We live in a distinctive world
\end{itemize}

\section{The Argument}\label{the-argument}

\begin{enumerate}
\def\labelenumi{\arabic{enumi}.}
\tightlist
\item
  At least one in ten nonsim populations will each create a thousand sim
  populations.

  \begin{enumerate}
  \def\labelenumii{\arabic{enumii}.}
  \tightlist
  \item
    It will never happen objections
  \end{enumerate}
\item
  If at least one in ten nonsim populations will each create a thousand
  sim populations, then at least 99 percent of intelligent beings are
  sims.
\item
  If at least 99 percent of intelligent beings are sims, we are probably
  sims.

  \begin{enumerate}
  \def\labelenumii{\arabic{enumii}.}
  \tightlist
  \item
    are we special?
  \end{enumerate}
\item
  Therefore, we are probably sims.
\end{enumerate}

\section{Premise 1 Objections}\label{premise-1-objections}

\begin{itemize}
\tightlist
\item
  Intelligent sims are impossible
\item
  Sims take too much computing power
\item
  Nonsims will die out before creating sims
\item
  Nonsims will choose not to create sims
\item
  More nonsims than sims will be created
\end{itemize}

\section{Premise 3, Are we special?}\label{premise-3-are-we-special}

\begin{quote}
Let's say that a sim sign is a feature that raises the probability that
a creature is a sim. More precisely, it is a feature that a sim is more
likely to have than a nonsim.
\end{quote}

Sim Signs: feature sims are more likely to have than nonsims

\begin{itemize}
\tightlist
\item
  Sims can't be conscious
\item
  Simulators will avoid creating conscious sims
\item
  Sims won't have minds like ours
\item
  Sims won't experience large universes
\end{itemize}

\begin{center}\rule{0.5\linewidth}{0.5pt}\end{center}

\begin{quote}
Stepping back: The potential nonsim signs we've considered, such as
consciousness and a large world, may decrease the probability that we're
in a simulation. At the same time, we need to weigh these against
potential sim signs, such as the fact that we seem to be early in the
universe, which may increase the probability that we're in a simulation.
\end{quote}

\chapter{Bostrom's Argument}\label{bostroms-argument}

\begin{enumerate}
\def\labelenumi{\arabic{enumi}.}
\tightlist
\item
  If there are no sim blockers, most humanlike beings are sims.
\item
  If most humanlike beings are sims, we are probably sims.
\item
  So: If there are no sim blockers, we are probably sims.
\end{enumerate}

\section{Premise 1}\label{premise-1}

Only on the assumption that there are no sim blockers, then \emph{most}
humanlike beings are sims.

\begin{itemize}
\tightlist
\item
  Does anything \emph{prevent} the creation of many humanlike sims?
\end{itemize}

\section{Premise 2, the Indifference
Principle}\label{premise-2-the-indifference-principle}

\begin{enumerate}
\def\labelenumi{\arabic{enumi}.}
\tightlist
\item
  If there are many beings with the same sort of experience as me, then
  I am equally likely to be any of those beings.
\item
  Therefore if 90 percent of beings with experiences like mine are sims,
  then I should be 90 percent confident that we are sims.
\item
  Simblockers: However, if conscious sims are impossible, then humanlike
  sims are impossible.
\item
  and if simulations with apparently large universes are rare, then
  humanlike sims are rare.
\item
  Therefore, either there are sim blockers or we are sims.
\end{enumerate}

\chapter{Chapter 6: Reality}\label{chapter-6-reality}

Consider simblockers

\begin{itemize}
\tightlist
\item
  Sims can't be conscious
\item
  Simulators will avoid creating conscious sims
\item
  Sims won't have minds like ours
\item
  Sims won't experience large universes
\end{itemize}

\section{What is Real?}\label{what-is-real}

Chalmer's View:

\begin{itemize}
\tightlist
\item
  Virtual Realism: virtual reality is genuine reality, . . . virtual
  objects are real and not an illusion.
\item
  Simulation Realism: objects in simulation around us are real and not
  illusion.
\item
  Virtual digitilism: objects in virtual reality are digital objects,
  structures of binary information.
\end{itemize}

\section{Defining Real}\label{defining-real}

\begin{itemize}
\tightlist
\item
  Reality: Everything that exists
\item
  Reality: World or Worlds
\item
  Reality: a Property
\end{itemize}

\begin{center}\rule{0.5\linewidth}{0.5pt}\end{center}

\subsection{Reality+}\label{reality}

\begin{itemize}
\tightlist
\item
  Reality contains many realities
\item
  Each of these realities are real
\end{itemize}

\begin{quote}
Or more mundanely: the cosmos (everything that exists) contains many
worlds (physical and virtual spaces), and the objects in those worlds
are real.
\end{quote}

\chapter{The Really Real}\label{the-really-real}

\begin{itemize}
\tightlist
\item
  Reality as existence
\item
  Reality as causal power
\item
  Reality as mind independence
\item
  Reality as non-illusoriness
\item
  Reality as genuineness
\end{itemize}

\chapter{Ch 7, Feb 7: Is God a Hacker in the next Universe
Up?}\label{ch-7-feb-7-is-god-a-hacker-in-the-next-universe-up}

\section{Is God a Hacker?}\label{is-god-a-hacker}

\begin{center}\rule{0.5\linewidth}{0.5pt}\end{center}

\subsection{What is God?}\label{what-is-god}

\begin{itemize}
\tightlist
\item
  Creator
\item
  All Powerful
\item
  All Knowing
\item
  All Good
\end{itemize}

\begin{center}\rule{0.5\linewidth}{0.5pt}\end{center}

\subsection{What about the Hacker?}\label{what-about-the-hacker}

\begin{itemize}
\tightlist
\item[$\square$]
  Creator?
\item[$\square$]
  Powerful?
\item[$\square$]
  Knows Stuff?
\item[$\square$]
  Good?
\end{itemize}

\begin{center}\rule{0.5\linewidth}{0.5pt}\end{center}

Important Distinction:

\begin{itemize}
\tightlist
\item
  Local vs.~Cosmic
\item
  Local Knowledge vs.~Global Knowledge
\item
  Local Power vs.~Global Power
\end{itemize}

\section{Proofs for the Existence of
God}\label{proofs-for-the-existence-of-god}

\begin{center}\rule{0.5\linewidth}{0.5pt}\end{center}

\subsection{Ontological Argument}\label{ontological-argument}

This is the a priori argument : prior to considering the existence of
the physical universe. This is reasoning without bringing in any
consideration of the existence of the universe or any part of it. This
is an argument considering the idea of god alone.

The argument is considered to be one of the most intriguing ever
devised. It took over 400 years for Philosophers to realize what its
actual flaws were. As an ``a priori'' argument, the Ontological Argument
tries to ``prove'' the existence of God by establishing the necessity of
God's existence through an explanation of the concept of existence or
necessary being .

\begin{center}\rule{0.5\linewidth}{0.5pt}\end{center}

VIEW:
\href{https://youtu.be/FmTsS5xFA6k?si=VYTUlDTrxcRIiji9}{Ontological
Argument}

VIEW:
\href{https://youtu.be/FmTsS5xFA6k?si=VYTUlDTrxcRIiji9}{Ontological
Argument and Anselm: Crash Course Philosophy \#9}

\begin{center}\rule{0.5\linewidth}{0.5pt}\end{center}

Anselm, Archbishop of Canterbury first set forth the Ontological
Argument in the eleventh century. This argument is the primary locus for
such philosophical problems as whether existence is a property and
whether or not the notion of necessary existence is intelligible. It is
also the only one of the traditional arguments that clearly leads to the
necessary properties of God, such as Omnipotence, Omniscience, etc.
Anselm's argument may be conceived as a ``reductiio ad absurdum''
argument. In such an argument, one begins with a supposition, which is
the contrary to what one is attempting to prove. Coupling the
supposition with various existing certain or self-evident assumption
will yield a contradiction in the end. This contradiction is what is
used to demonstrate that the contrary of the original supposition is
true.

\begin{center}\rule{0.5\linewidth}{0.5pt}\end{center}

Two Forms:

Form 1:

Premises:

(1.a.) Anselm- the supreme being- that being greater than which none can
be conceived (gcb)

(1.b.) the gcb must be conceived of as existing in reality and not just
in the mind or else the gcb is not that being greater than which none
can be conceived.

Suppose (S) that the greatest conceivable being (GCB) exists in the mind
alone and not in reality(gcb1).\\

Then the greatest conceivable being would not be the greatest
conceivable being because one could think of a being like (gcb1) but
think of the gcb as existing in reality (gcb2) and not just in the
mind.\\

So, gcb1 would not be the GCB but gcb2 would be.

Conclusion:

Thus to think of the GCB is to think of the gcb2, i.e.~a being that
exists in reality and not just in the mind.

\begin{center}\rule{0.5\linewidth}{0.5pt}\end{center}

Form 2: God as Necessary Being

Premises:

\begin{enumerate}
\def\labelenumi{(\alph{enumi})}
\tightlist
\item
  God is either a necessary being or a contingent being.\\
\item
  There is nothing contradictory about god being a necessary being\\
\item
  So, it is possible that god exists as a necessary being.\\
\item
  So if it is possible that God is a necessary being then God exists.\\
\item
  Because God is not a contingent being.\\
\end{enumerate}

Conclusion:

God must exist as a necessary being.

Notes on the Ontological arguments of Anselm and Descartes

Anselm begins by defining the most central term in his argument - God.
Without asserting that God exists, Anselm asks what is it that we mean
when we refer to the idea of ``God.'' When we speak of a God, Anselm
implies, we are speaking of the most supreme being. That is, let ``god''
= ``something than which nothing greater can be thought.'' Anselm's
definition of God might sound confusing upon first hearing it, but he is
simply restating our intuitive understanding of what is meant by the
concept ``God.'' Thus, for the purpose of this argument let ``God'' =
``a being than which nothing greater can be conceived.''

Within your understanding, then, you possess the concept of God. As a
non-believer, you might argue that you have a concept of unicorn (after
all, it is the shared concept that allows us to discuss such a thing)
but the concept is simply an idea of a thing. After all, we understand
what a unicorn is but we do not believe that they exist. Anselm would
agree.

\begin{center}\rule{0.5\linewidth}{0.5pt}\end{center}

Two key points have been made thus far:

\begin{enumerate}
\def\labelenumi{(\arabic{enumi})}
\item
  When we speak of God (whether we are asserting God is or God is not),
  we are contemplating an entity whom can be defined as ``a being which
  nothing greater can be conceived.'';
\item
  When we speak of God (either as believer or non-believer), we have an
  intra-mental understanding of that concept, i.e.~the idea is within
  our understanding.
\end{enumerate}

Anselm continues by examining the difference between that which exists
in the mind and that which exists both in the mind and outside of the
mind as well. What is being asked here is: Is it greater to exist in the
mind alone or in the mind and in reality (or outside of the mind)?
Anselm asks you to consider the painter, e.g.~define which is greater:
the reality of a painting as it exists in the mind of an artist, or that
same painting existing in the mind of that same artist and as a physical
piece of art. Anselm contends that the painting, existing both within
the mind of the artist and as a real piece of art, is greater than the
mere intra-mental conception of the work. Let me offer a real-world
example: If someone were to offer you a dollar, but you had to choose
between the dollar that exists within their mind or the dollar that
exists both in their mind and in reality, which dollar would you choose?
Are you sure\ldots{}

\begin{center}\rule{0.5\linewidth}{0.5pt}\end{center}

At this point, we have a third key point established:

\begin{enumerate}
\def\labelenumi{(\arabic{enumi})}
\setcounter{enumi}{2}
\tightlist
\item
  It is greater to exist in the mind and in reality, then to exist in
  the mind alone.
\end{enumerate}

Have you figured out where Anselm is going with this argument?

\begin{enumerate}
\def\labelenumi{(\Alph{enumi})}
\tightlist
\item
  If God is that than greater which cannot be conceived (established in
  \#1 above);
\item
  And since it is greater to exist in the mind and in reality than in
  the mind alone (established in \#3 above);
\item
  Then God must exist both in the mind (established in \#2 above) and in
  reality;
\item
  In short, God must be. God is not merely an intra-mental concept but
  an extra-mental reality as well.
\end{enumerate}

But why? Because if God is truly that than greater which cannot be
conceived, it follows that God must exist both in the mind and in
reality. If God did not exist in reality as well as our understanding,
then we could conceive of a greater being i.e.~a being that does exist
extramentally and intramentally. But, by definition, there can be no
greater being. Thus, there must be a corresponding extra-mental reality
to our intra-mental conception of God. God's existence outside of our
understanding is logically necessary.

Sometimes, Anselm's argument is presented as a Reductio Ad Absurdum
(RAA). In an RAA, you reduce to absurdity the antithesis of your view.
Since the antithesis is absurd, your view must be correct. Anselm's
argument would look something like this:

\begin{center}\rule{0.5\linewidth}{0.5pt}\end{center}

\begin{enumerate}
\def\labelenumi{\arabic{enumi}.}
\item
  Either {[}God exists{]} or {[}God does not exist{]}.
\item
  Assume {[}God does not exist{]} (the antithesis of Anselm's position)
\item
  If {[}God does not exist{]} (but exists only as an intra-mental
  concept), then that being which nothing greater which can be
  conceived, is a being which a greater being can be conceived. This is
  a logical impossibility (remember criterion \#3);
\item
  Therefore, {[}God does not exist{]} is incorrect;
\end{enumerate}

Conclusion:

\begin{enumerate}
\def\labelenumi{\arabic{enumi}.}
\setcounter{enumi}{4}
\tightlist
\item
  Therefore {[}God exists{]}.
\end{enumerate}

\begin{center}\rule{0.5\linewidth}{0.5pt}\end{center}

\subsection{Clarifications:}\label{clarifications}

\begin{itemize}
\tightlist
\item
  The argument is not that ``If you believe that god exists then god
  exists''.
\item
  That would be too ridiculous to ask anyone to accept that if you
  believe that X exists and is real then X exists and is real.
\item
  The ontological argument does not ask a person to assume that there is
  a deity or even a GCB.
\end{itemize}

It asks anyone at all to simply THINK of the deity as the GREATEST
CONCEIVABLE BEING and then it indicates that a being that exists in
reality (outside of the mind) is greater than one that is just in the
mind (imagination). So, the conclusion is that if you think of the GCB
you must THINK that the GCB exists not just in your thinking (mind) but
in reality (outside of your mind) as well.

It is greater to think of a being existing outside of the mind as well
as in the mind so if you think of the GCB you must THINK THAT the GCB
exists not just inside of the mind (imagination) but outside of the mind
as well (in reality).

Look at it this way: Anselm invites people to think about a certain
conception of the deity,i.e., that of the GCB. What Anselm did was to
place into the concept itself the idea that the being must exist outside
of the mind and in the realm of the real and not just inside the mind in
the realm of imagination. So you THINK of the GCB and what are you doing
when you do that? You must think that the GCB exists outside of the mind
and in the realm of the real and not just inside the mind in the realm
of imagination. Why must you think that? Because it you did not think
that, then you would not be thinking of the GCB as defined by Anselm.

It is like this: Think of a triangle. If you do you must think of a
three sided figure lying on a plane with three angles adding up to 180
degrees. Why? Because if you are not thinking of a three sided figure
lying on a plane with three angles adding up to 180 degrees then you are
not thinking of a triangle. So IF you are to THINK of a triangle you
must THINK of a three sided figure lying on a plane with three angles
adding up to 180 degrees.

If you are to THINK of a GCB you must THINK that the being must exist
outside of the mind and in the realm of the real and not just inside the
mind in the realm of imagination. Why? Because if you are not thinking
that the being must exist outside of the mind and in the realm of the
real and not just inside the mind in the realm of imagination then you
are not thinking of the GCB.

In all of this it is only thinking. Anselm proved what must be thought
about the GCB given how the GCB was defined and not whether the GCB
actually exists.

\begin{center}\rule{0.5\linewidth}{0.5pt}\end{center}

A variation of this argument by Alvin Plantinga exists. It is known as
the Modal Version of the Ontological Argument:

\begin{enumerate}
\def\labelenumi{\arabic{enumi}.}
\item
  To say that there is possibly a God is to say that there is a possible
  world in which God exists.
\item
  To say that God necessarily exists is to say that God exists in every
  possible world.
\item
  God is necessarily perfect (i.e.~maximally excellent)
\item
  Since God is necessarily perfect, he is perfect in every possible
  world.
\item
  If God is perfect in every possible world, he must exist in every
  possible world, therefore God exists.
\item
  God is also maximally great. To be maximally great is to be perfect in
  every possible world.
\item
  Therefore: ``it is possible that there is a God,'' means that there is
  a possible which contains God, that God is maximally great, and the
  God exists in every possible world and is consequently necessary.
\item
  God's existence is at least possible.
\item
  Therefore: as per item seven, God exists.
\end{enumerate}

\begin{center}\rule{0.5\linewidth}{0.5pt}\end{center}

Rene Descartes, 1596 - 1650, is also credited with formulating a version
of the ontological argument. One possible presentation of the Cartesian
argument is as follows:

\begin{enumerate}
\def\labelenumi{\arabic{enumi}.}
\item
  If there is a God it is a perfect being;
\item
  A perfect being possesses all possible perfections;
\item
  Existence is a perfection;
\item
  Therefore, God necessarily possesses the quality of existence. Simply,
  God exists.
\end{enumerate}

\begin{center}\rule{0.5\linewidth}{0.5pt}\end{center}

The actual texts:

\href{https://sourcebooks.fordham.edu/basis/anselm-intro.asp}{Anselm's
Philosophy}

Anselm's Argument

\href{https://sourcebooks.fordham.edu/basis/anselm-monologium.asp}{Monologium}

\href{https://sourcebooks.fordham.edu/basis/anselm-proslogium.asp}{Proslogium}

\href{https://sourcebooks.fordham.edu/basis/anselm-gaunilo.asp}{Guanilo's
Response and Anselm's response to Guanilo}

\begin{center}\rule{0.5\linewidth}{0.5pt}\end{center}

PROBLEMS:

The problem with the ontological argument is NOT

\begin{enumerate}
\def\labelenumi{\arabic{enumi})}
\item
  that some people refuse to think of the GCB or
\item
  that some people have a resistance to a belief in a deity
\item
  that some people just refuse to accept the deity
\end{enumerate}

NO NO NO the problem with the Argument is that it has FLAWS. It has a
LOGICAL MISTAKE in it.

What is that error in the argument???

\section{PROBLEM:}\label{problem}

Conclusion of the argument is : Thus, to think of the GCB is to think of
the gcb2, i.e.~a being that exists in reality and not just in the mind\\

\begin{itemize}
\item
  Immanuel Kant noticed that to think of the GCB is to think of the
  gcb2, i.e.~a being that exists in reality and not just in the mind\\
\item
  BUT to think of the gcb2 as a being that exists in reality and not
  just in the mind, does not prove that the gcb2 does actually exist in
  reality ONLY that a person MUST THINK that the gcb2 does actually
  exist in reality
\item
  But for Kant and many after him , the notion of ``Existence'' is not a
  predicate: You cannot include it within the idea of the thing itself.
  You cannot think anything into existence by including existence as a
  property of that thing.
\end{itemize}

\chapter{Counter Arguments to
Anselm:}\label{counter-arguments-to-anselm}

\section{I. The Most Perfect Island}\label{i.-the-most-perfect-island}

Gaunilon, a contemporary of Anselm, had two major criticisms of the
ontological argument.

\begin{itemize}
\item
  First: If by ``God'' we do mean ``that than greater which can not be
  conceived,'' then the concept is meaningless for us. We can not
  understand, in any meaningful way, what exactly is meant by such
  words. The reality behind the term is completely transcedent to the
  human knower;
\item
  Second: Even if we grant that the concept of God as ``that than
  greater which can not be conceived'' exists in the understanding,
  there is no reason to believe that the concept necessitates the
  extra-mental reality of God. After all, I can imagine the most perfect
  island, glorious in every detail, but there is nothing about my
  understanding of the island that forces us to admit the island exists.
\end{itemize}

\begin{center}\rule{0.5\linewidth}{0.5pt}\end{center}

\section{II. Existence is not a
predicate}\label{ii.-existence-is-not-a-predicate}

Immanuel Kant (1724 - 1804), offered what many believe to be a damning
critique of Anselm's ontological argument.

\begin{itemize}
\tightlist
\item
  Let us return to our discussion of unicorns and God.
\item
  Anselm has argued that there exists a difference between the concept
  of ``unicorn'' as it exists intra-mentally and extra-mentally.
\item
  If we claim that the ``unicorn'' is, we are somehow adding to the
  concept. We are endowing the concept with an additional predicate,
  i.e.~the quality that it is.
\end{itemize}

The point of Anselm's argument is that the predicate of existence can be
demonstrated for the concept of ``God.''

Kant does not agree with Anselm's treatment of existence as a predicate.
The concept of ``unicorn'' is not changed in any way if we claim that it
is. Nor is the concept damaged if we claim that unicorns are not.
According to Kant,``\ldots we do not make the least addition to the
thing when we further declare that this thing is.'' If existence is not
a predicate, then Anselm's argument has not demonstrated any meaningful
information.

Kant thought that, while the concept of a supreme being was useful, it
was only an idea, which in and of itself could not help us in our
determining the correctness of the concept. While it was a possibility,
he felt that the ``a priori'' stance of the argument it would be
necessary to buttress it with experience.

For Kant what Anselm did was to prove that humans MUST THINK THAT a
deity exists in reality and not just in the mind as an idea as the GCB
but that does not mean that the GCB actually does exist in reality. The
idea of the GCB exists and the idea of the GCB as an actual being does
exist but the reality or actuality of the GCB is not established based
on the thoughts alone.

\section{Think of three situations:}\label{think-of-three-situations}

\begin{enumerate}
\def\labelenumi{\arabic{enumi}.}
\item
  You go home and look at the top of your dresser. You could use some
  money and as you look there you imagine seeing ten ten dollar bills.
\item
  You go home and look at the top of your dresser. You could use some
  money and as you look there you see ten MONOPOLY ten dollar bills.
\item
  You go home and look at the top of your dresser. You could use some
  money and as you look there you seeing ten real ten dollar bills.
\end{enumerate}

Which of the three is the greatest or best situation? \#3 is.

But just thinking about \#3 does not actually add any money to your
total amount.

This is Kant's point.

Thinking about the GCB logically entails THINKING that the GCB must
exist in reality and not just in the imagination. But thinking about the
GCB as existing in reality and not just in the imagination does not
prove that the GCB actually does exist in reality and not just in the
imagination. It is just an idea about what exists.

\section{III. The Greatest Conceivable EVIL
Being.}\label{iii.-the-greatest-conceivable-evil-being.}

As an ``a priori'' argument, the Ontological Argument tries to ``prove''
the existence of God by establishing the necessity of God's existence
through an explanation of the concept of existence or necessary being.
As this criticism of the Ontological Argument shows, the same arguments
used to prove an all-powerful god, could be used to prove an
all-powerful devil. Since there could not exist two all-powerful beings
(one's power must be subordinate to the other), this is an example of
one of the weaknesses in this type of theorizing. Furthermore, the
concept of necessary existence, by using Anselm's second argument,
allows us to ``define'' other things into existence.

The argument could prove the existence of that being more EVIL than
which no other can be conceived just as easily as it supposedly proves
the existence of the being that is the greatest conceivable being.

Think of a being that is the most evil being that can be conceived. That
being must be conceived of as existing in reality and not just in the
mind or it wouldn't be the most evil being which can be conceived for a
being that does not exist in reality is not evil at all.

\section{IV. Empiricist Critique}\label{iv.-empiricist-critique}

Aquinas, 1225 - 1274, once declared the official philosopher of the
Catholic Church, built his objection to the ontological argument on
epistemological grounds.

\begin{itemize}
\item
  Epistemology is the study of knowledge. It is a branch of philosophy
  that seeks to answer such questions as: What is knowledge?; What is
  truth?; How does knowing occur?; et cetera. Aquinas is known as an
  empiricist. Empiricists claim that knowledge comes from sense
  experience. Aquinas wrote: ``Nothing is in the intellect which was not
  first in the senses.''
\item
  Within Thomas' empiricism, we can not reason or infer the existence of
  God from a studying of the definition of God. We can know God only
  indirectly, through our experiencing of God as Cause to that which we
  experience in the natural world. We can not assail the heavens with
  our reason; we can only know God as the Necessary Cause of all that we
  observe.
\item
  Alvin Plantiga offers a counter argument to the counter arguments that
  at least establishes the rational acceptability of theism as it
  appears to support the idea that it is possible that the greatest
  conceivable being does exist.
\end{itemize}

\begin{center}\rule{0.5\linewidth}{0.5pt}\end{center}

\subsection{Other Philosophers and their
Critiques:}\label{other-philosophers-and-their-critiques}

\begin{enumerate}
\def\labelenumi{(\alph{enumi})}
\item
  \href{https://www.qcc.cuny.edu/socialSciences/ppecorino/INTRO_TEXT/Chapter\%203\%20Religion/Ontological.htm\#DESCARTES}{René
  Descartes, from The Philosophy of Descartes in Extracts from His
  Writings. H. A. P. Torrey. New York, 1892. P. 161 et seq.}
\item
  \href{https://www.qcc.cuny.edu/socialSciences/ppecorino/INTRO_TEXT/Chapter\%203\%20Religion/Ontological.htm\#SPINOZA}{Benedict
  Spinoza, from The Chief Works of Benedict de Spinoza. Translated by
  R.H.M.Elwes. London, 1848. VoI. II., P. 51 at seq.}
\item
  \href{https://www.qcc.cuny.edu/socialSciences/ppecorino/INTRO_TEXT/Chapter\%203\%20Religion/Ontological.htm\#LOCKE}{John
  Locke, from An Essay Concerning Human Understanding. London: Ward,
  Lock, Co.~P. 529 et seq.}
\item
  \href{https://www.qcc.cuny.edu/socialSciences/ppecorino/INTRO_TEXT/Chapter\%203\%20Religion/Ontological.htm\#LEIBNIZ}{Gottfried
  W. Leibniz, from New Essays Concerning Human Understanding. Translated
  by A.G. Langley. New York, 1896. P. 502 at seq.}
\item
  \href{https://www.qcc.cuny.edu/socialSciences/ppecorino/INTRO_TEXT/Chapter\%203\%20Religion/Ontological.htm\#KANT}{Immanuel
  Kant, from Critique of Pure Reason. Translated by F. Max Muller. New
  York, 1896. P-483 et seq.}
\item
  \href{https://www.qcc.cuny.edu/socialSciences/ppecorino/INTRO_TEXT/Chapter\%203\%20Religion/Ontological.htm\#HEGEL}{Georg
  W.F. Hegel, from Lectures on the History of Philosophy. Translated by
  E. S. Haldane and F.H. Simson. London, 1896. Vol. III., p.~62 et seg.}
\item
  \href{https://www.qcc.cuny.edu/socialSciences/ppecorino/INTRO_TEXT/Chapter\%203\%20Religion/Ontological.htm\#J.\%20A.\%20DORNER}{J.
  A. Dorner from A System of Christian Doctrine. Translated by A. Cave
  and J. S. Banks, Edinburgh, 1880. Vol. I., p.~216 et seq}
\item
  \href{https://www.qcc.cuny.edu/socialSciences/ppecorino/INTRO_TEXT/Chapter\%203\%20Religion/Ontological.htm\#LOTZE}{Lotze,
  Microcosmus. Translated by E. Hamilton and E. E. C. Jones. Edinburgh,
  1887. Vol. II., p.~669 et seq.}
\item
  \href{https://www.qcc.cuny.edu/socialSciences/ppecorino/INTRO_TEXT/Chapter\%203\%20Religion/Ontological.htm\#ROBERT\%20FLINT}{Robert
  Flint, from Theism. New York, 1893. Seventh edition. P. 278 et seq.}
\end{enumerate}

\begin{center}\rule{0.5\linewidth}{0.5pt}\end{center}

View also \href{http://www.youtube.com/watch?v=iRulK_ePLKM}{Debunking
the Teleological},
\href{http://www.youtube.com/watch?v=iRulK_ePLKM}{Cosmological}, and
\href{http://www.youtube.com/watch?v=iRulK_ePLKM}{Ontological Arguments
for the Existence of God}

\href{http://www.fordham.edu/halsall/basis/anselm-critics.html}{Read the
critiques of the Ontological Argument}

\begin{center}\rule{0.5\linewidth}{0.5pt}\end{center}

\subsection{Concluding Summary:}\label{concluding-summary}

\begin{enumerate}
\def\labelenumi{\arabic{enumi}.}
\tightlist
\item
  What it does prove:
\end{enumerate}

\begin{enumerate}
\def\labelenumi{(\alph{enumi})}
\tightlist
\item
  Anselm proves that if you think of the GCB you must THINK that it
  exists.
\item
  Descartes proves that if you conceive of an ALL PERFECT being you must
  CONCEIVE (THINK) of that being as existing.
\end{enumerate}

\begin{enumerate}
\def\labelenumi{\arabic{enumi}.}
\setcounter{enumi}{1}
\item
  Kant points out that even though you must THINK that it exists does
  not mean that it does exist. Existence is not something we can know
  from the mere idea itself. It is not known as a predicate of a
  subject. Independent confirmation through experience is needed.
\item
  The argument does give some support to those who are already
  believers. It has variations that establish the possibility of the
  existence of such a being.
\item
  The argument will not convert the non-believer into a believer.
\end{enumerate}

\begin{center}\rule{0.5\linewidth}{0.5pt}\end{center}

\subsection{Outcome Assessment}\label{outcome-assessment}

This argument or proof does not establish the actual existence of a
supernatural deity. It attempts to define a being into existence and
that is not rationally legitimate. While the argument can not be used to
convert a non-believer to a believer, the faults in the argument do not
prove that there is no god. The Burden of Proof demands that the
positive claim that there is a supernatural deity be established by
reason and evidence and this argument does not meet that standard. The
believer in god can use the argument to establish the mere logical
possibility that there is a supernatural deity or at least that it is
not irrational to believe in the possibility that there is such a being.
The argument does not establish any degree of probability at all.

\begin{center}\rule{0.5\linewidth}{0.5pt}\end{center}

OUTCOME:

The Argument:

Premises

\begin{itemize}
\tightlist
\item
  Suppose (S) that the greatest conceivable being (GCB) exists in the
  mind alone and not in reality(gcb1).
\item
  Then the greatest conceivable being would not be the greatest
  conceivable being because one could think of a being like (gcb1) but
  think of the gcb as existing in reality (gcb2) and not just in the
  mind.
\item
  So, gcb1 would not be the GCB but gcb2 would be.
\item
  Thus to think of the GCB is to think of the gcb2, i.e.~a being that
  exists in reality and not just in the mind.
\end{itemize}

\begin{center}\rule{0.5\linewidth}{0.5pt}\end{center}

\subsection{Conclusion: The GCB ( Deity)
exists}\label{conclusion-the-gcb-deity-exists}

Problem with argument:

\begin{enumerate}
\def\labelenumi{\arabic{enumi}.}
\item
  \_\_\_\_Premises are false
\item
  \_\_\_\_Premises are irrelevant
\item
  \_\_\_\_Premises Contain the Conclusion --Circular Reasoning
\item
  \_\_X\_\_Premises are inadequate to support the conclusion
\item
  \_\_\_\_Alternative arguments exist with equal or greater support
\end{enumerate}

\begin{center}\rule{0.5\linewidth}{0.5pt}\end{center}

\begin{itemize}
\tightlist
\item
  This argument or proof has flaws in it and would not convince a
  rational person to accept its conclusion.
\item
  This is not because someone who does not believe in a deity will
  simply refuse to accept based on emotions or past history but because
  it is not rationally compelling of acceptance of its conclusion.
\end{itemize}

\begin{center}\rule{0.5\linewidth}{0.5pt}\end{center}

It would be a mistake in thinking, a violation of logic and a fallacy to
think that because this argument or attempt to prove that there is a
deity of some type does not work or has flaws that the opposite
conclusion must be true, namely that there is no deity of any type. The
error is known as the fallacy ``argumentum ad ignoratio'' or the appeal
to ignorance. It is the mistake in thinking that if an argument cannot
prove a proposition or claim P is true then P must be false. OR if you
cannot prove that P is false then P must be true. It is a mistake to
think that way., a logical error.

Proceed to the next section.

Creative Commons License Introduction to Philosophy by Philip A.
Pecorino is licensed under a Creative Commons
Attribution-NonCommercial-NoDerivs 3.0 Unported License. Return to:
Table of Contents for the Online Textbook

\chapter{Assignments}\label{assignments-1}

\chapter{Weeklies}\label{weeklies-1}

\textbf{\emph{Weekly Reflections (32 points):}} These will be done each
week in class. Each reflection is worth a possible total of 8 points.
There are 12 possible reflections. I will only grade 8 or your best
reflections for a possible total of 32 points.

\chapter{Quizzes}\label{quizzes-1}

\textbf{\emph{Quiz 1 \& 2 (80 points):}} Each Quiz is worth a possible
total of 10 points per category, times 4 categories equals 40 points
times 2 Quizzes.

\chapter{Critical Reflections}\label{critical-reflections}

\textbf{\emph{Critical Reflection 1 \& 2 (160 points):}} Each reflective
analysis is worth a possible total of 20 points per category, times 4
categories equals 80 points times 2 reflection pieces.

\chapter{D\&D Campaigns}\label{dd-campaigns}

The D\&D assignment is not exactly like D\&D though Ive tried to keep it
as close as possible. The idea is to encourage students to talk through
the philosophical issues we read about.

\textbf{\emph{Presentation (96 points):}} Your grade in this area
depends on your preparation and participation in the DND class meetings.
\href{https://drive.google.com/drive/folders/1ZAXxdpvv5jzylj8_18jXUTI68o9MLrXw?usp=sharing}{We
will use the dnd adventure sheet linked here to gage participation}.

\part{Philosophy of Law}

\chapter{Syllabus}\label{syllabus-2}

\section{Course Description}\label{course-description}

We will examine theories which attempt to provide answers to such
questions as: What is a law? What makes a law valid or binding? Does one
always have a moral obligation to obey the law? What gives society the
right to punish people whose actions are harmful only to the individual,
him/herself (paternalism)? Does society have the right to harm those who
break the law (criminal punishment)? Should attempted crimes be punished
less severely than completed crimes? We will not spend much time
discussing specific public policy issues such as: Whether raising the
drinking age saves lives, whether smoking marijuana is a victimless
crime (whether it harms others), or whether the death penalty is a
deterrent. We will not discuss what the law is; instead, we will discuss
what the law ought to be.

\subsection{📑 Course Brief}\label{course-brief}

\textbf{Focus:}

\begin{enumerate}
\def\labelenumi{\arabic{enumi}.}
\tightlist
\item
  Developing critical reasoning skills.
\item
  Discovering some of the fundamental philosophical ideas in law.
\end{enumerate}

\textbf{Text:} Feinberg, Joel, and Jules L. Coleman. 2008. Philosophy of
Law / {[}Edited by{]} Joel Feinberg, Jules Coleman. 8th
ed.~Thomson/Wadsworth.

🎯 Learning Objectives

\begin{itemize}
\tightlist
\item
  What is a law?
\item
  What makes a law valid or binding?
\item
  Do we have a moral obligation to obey the law?
\item
  What gives society the right to punish people whose actions are
  harmful only to the individual, him/herself (paternalism)?
\item
  Does society have the right to harm those who break the law (criminal
  punishment)?
\item
  Should attempted crimes be punished less severely than completed
  crimes?
\end{itemize}

What is not covered

Public policy issues such as:

\begin{itemize}
\tightlist
\item
  Whether raising the drinking age saves lives
\item
  Wehther smoking marijuana is a victimless crime (whether it harms
  others)
\item
  Whether the death penalty is a deterrent.
\end{itemize}

We do not discuss what the law is; instead, we will discuss what the law
ought to be.

\section{The Speluncean Explorers}\label{the-speluncean-explorers}

\textbf{\emph{The Speluncean Explorers}}

\section{Course Convenor}\label{course-convenor-1}

\begin{center}
\includegraphics{figures/people/professor_octopus.png}
\end{center}
\textbf{\hyperref[]{Reynolds}} Dr.~Monty Reynolds \hyperref[]{Stetson
University} 📧 \href{mailto:}{mreynolds1@stetson.edu}

Office Hours:

\begin{itemize}
\tightlist
\item
  When:

  \begin{itemize}
  \tightlist
  \item
    Tuesday: 1:00--3:00 PM
  \item
    Thursday: 1:00--3:00 PM
  \end{itemize}
\item
  Where: Elizabeth Hall 104
\item
  How to book: Drop in, email, or book via
  \href{https://outlook.office.com/bookwithme/user/59570715559b43fdabc651a89a3ed839@stetson.edu?anonymous&ismsaljsauthenabled&ep=plink}{Microsoft
  Bookings}
\end{itemize}

\section{Course Information}\label{course-information-1}

\subsection{Required Texts:}\label{required-texts-1}

\href{https://www.amazon.com/Philosophy-text-Eighth-Feinberg-Coleman/dp/B004IS2MP6/ref=monarch_sidesheet_title}{Feinberg,
Joel, and Jules L. Coleman. 2008. Philosophy of Law / Edited by Joel
Feinberg, Jules Coleman. 8th ed.~Thomson/Wadsworth.}

Required: Handouts

\subsection{Expectations:}\label{expectations-1}

Come prepared to engage with assigned readings in class, referencing
specific passages as prompted by the instructor.

Bring physical or digital copies of readings to class for annotation and
short reflections.

Submit all assignments via Canvas by the due date.

Active participation and regular attendance are essential for success.

\subsection{Late Assignment Policy}\label{late-assignment-policy}

Penalty: 10\% deduction per day late.

Makeup Process: Email the instructor within 48 hours to arrange an
extension. No credit if not submitted by agreed date.

\subsection{Success: Active participation, timely submissions, and
attendance are
key.}\label{success-active-participation-timely-submissions-and-attendance-are-key.}

\section{Grading Calculation}\label{grading-calculation-1}

\textbf{\emph{Weekly Reflections:}} Each reflection is worth a possible
total of 8 points. There are 12 possible reflections. I will only grade
8 or your best reflections.

\textbf{\emph{Reflective Analysis 1 \& 2 (160 points):}} Each reflective
analysis is worth a possible total of 20 points per category, times 4
categories equals 80 points times 2 reflection pieces.

\textbf{\emph{Essay 1 \& 2 (80 points):}} Each Quiz is worth a possible
total of 10 points per category, times 4 categories equals 40 points
times 2 Quizzes.

\textbf{\emph{Presentation:}} Your grade in this area depends on your
preparation and participation in the DND class meetings.
\href{https://drive.google.com/drive/folders/1ZAXxdpvv5jzylj8_18jXUTI68o9MLrXw?usp=sharing}{We
will use the dnd adventure sheet linked here to quantify participation}.

Students will be evaluated based on a total of 640 points, with the
final grade determined by the percentage of points earned. The
components are as follows:

\subsubsection{\texorpdfstring{For more detailed information regarding
grades, \href{./weeks/rubric.qmd}{see the rubric
here}}{For more detailed information regarding grades, see the rubric here}}\label{for-more-detailed-information-regarding-grades-see-the-rubric-here}

\subsubsection{\texorpdfstring{For more detailed information regarding
the DND Adventures, \href{./weeks/dnd-campaign.qmd}{see the DND addendum
here}}{For more detailed information regarding the DND Adventures, see the DND addendum here}}\label{for-more-detailed-information-regarding-the-dnd-adventures-see-the-dnd-addendum-here}

\subsection{Attendance (5\% of final grade, 32
points):}\label{attendance-5-of-final-grade-32-points-1}

\begin{itemize}
\tightlist
\item
  Based on unexcused absences throughout the semester.
\end{itemize}

Grading Scale (equal increments of 8 points):

\begin{longtable}[]{@{}lll@{}}
\toprule\noalign{}
\endhead
\bottomrule\noalign{}
\endlastfoot
0--1 & unexcused absences: & 32 points \\
2 & unexcused absences: & 24 points \\
3--4 & unexcused absences: & 16 points \\
5--6 & unexcused absences: & 8 points \\
6 & unexcused absences: & 0 points \\
\end{longtable}

\section{Course Schedule}\label{course-schedule-1}

Here is the syllabus formatted as a clean Markdown table using only the
vertical line \texttt{\textbar{}} as the separator (with a properly
spanned title row):

\begin{longtable}[]{@{}
  >{\raggedright\arraybackslash}p{(\columnwidth - 4\tabcolsep) * \real{0.3333}}
  >{\raggedright\arraybackslash}p{(\columnwidth - 4\tabcolsep) * \real{0.3333}}
  >{\raggedright\arraybackslash}p{(\columnwidth - 4\tabcolsep) * \real{0.3333}}@{}}
\toprule\noalign{}
\begin{minipage}[b]{\linewidth}\raggedright
Day
\end{minipage} & \begin{minipage}[b]{\linewidth}\raggedright
Date
\end{minipage} & \begin{minipage}[b]{\linewidth}\raggedright
Discussion
\end{minipage} \\
\midrule\noalign{}
\endhead
\bottomrule\noalign{}
\endlastfoot
Tuesday & January 13 & Introduction to Philosophy of Law and
Organization of Class \\
Thursday & January 15 & Fuller, ``The Case of the Speluncean Explorers''
(Handout - Canvas) \\
Tuesday & January 20 & Kretzmann, ``Lex Iniusta Non Est Lex: Laws on
Trial'' (Handout -- Canvas) \\
Thursday & January 22 & Bentham, ``The Principles of Morals and
Legislation'' (Handout, pages 224-248 -D2L) \\
Tuesday & January 27 & Bentham (Continued) \\
Thursday & January 29 & Hart, ``Law as the Union of Primary and
Secondary Rules'' (crc, 53) \\
Tuesday & February 3 & Hart (Continued) \\
Thursday & February 5 & Dworkin, ``The Model of Rules'' (148) \\
Tuesday & February 10 & Dworkin, ``Integrity in Law'' (169) \\
Thursday & February 12 & Holmes, ``The Path of the Law'' (197); Frank,
``Legal Realism'' (205) \\
Tuesday & February 17 & Critical Legal Studies (Handout -- D2L) \\
Thursday & February 19 & Plato, Crito (Handout -- D2L) \\
Tuesday & February 24 & Crito (Continued) \\
Thursday & February 26 & Quiz 1 \\
Tuesday & March 3 & Spring Break \\
Thursday & March 5 & Spring Break \\
Tuesday & March 10 & Hart, ``Postscript: Responsibility and
Retribution'' (317) \\
Thursday & March 12 & Duff, ``Choice, Character, and Action'' (328) \\
Tuesday & March 17 & Parker, ``Blame, Punishment and the Role of
Result'' (Handout -- D2L) \\
Thursday & March 19 & Spring Break \\
Tuesday & March 24 & Spring Break \\
Thursday & March 26 & Self Defense (Handout -- D2L) \\
Tuesday & March 31 & Self Defense (Continued) \\
Thursday & April 2 & Excuse, Justification, Subjective/Objective
Liability (Handout -- D2L) \\
Tuesday & April 7 & Mill, ``On Liberty'' (258) \\
Thursday & April 9 & Dworkin, ``Paternalism'' (271); Griswold v.
Connecticut (summary on www) \\
Tuesday & April 14 & Devlin, ``Morals and the Criminal Law'' (283) \\
Thursday & April 16 & ``The Moral Significance of Terrorism'' (Handout
-- D2L) \\
Tuesday & April 21 & Dershowitz, ``Should the Ticking Bomb Terrorist be
Tortured'' (Paper Due at End of Class) \\
Thursday & April 23 & Bork, ``The Original Understanding'' (Handout --
D2L) \\
Tuesday & April 28 & Bork (Continued), last day of classes \\
Thursday & April 30 & Final exams, final paper due, TBD \\
Tuesday & May 3 & Final exams TBD \\
\end{longtable}

\section{Grading Calculation}\label{grading-calculation-2}

Total Points: 640 points, distributed as above.

\begin{itemize}
\tightlist
\item
  Formula: Raw scores summed across sheets; final percentage = SUM(all
  assignment points)/640 * 100.
\end{itemize}

\textbf{\emph{Letter Grade:}}

Final grade uses the following scale (no rounding):

\begin{longtable}[]{@{}ll@{}}
\toprule\noalign{}
Letter Grade & Percentage \\
\midrule\noalign{}
\endhead
\bottomrule\noalign{}
\endlastfoot
A+ & 97--100\% \\
A & 93--96\% \\
A- & 90--92\% \\
B+ & 87--89\% \\
B & 83--86\% \\
B- & 80--82\% \\
C+ & 77--79\% \\
C & 73--76\% \\
C- & 70--72\% \\
D+ & 67--69\% \\
D & 63--66\% \\
D- & 60--62\% \\
F & \textless60\% \\
\end{longtable}

\subsection{Example: Perfect scores = 32 + 160 + 160 + 80 + 80 + 96 + 32
= 640 points (100\%,
A+).}\label{example-perfect-scores-32-160-160-80-80-96-32-640-points-100-a.}

\section{University Supports and
Policies}\label{university-supports-and-policies-1}

\chapter{Lectures}\label{lectures-2}

\section{Week 1}\label{week-1}

\section{Week 2}\label{week-2}

\section{Week 3}\label{week-3}

\section{Week 4}\label{week-4}

\section{Week 5}\label{week-5}

\section{Oliver Wendell Holmes}\label{oliver-wendell-holmes}

\begin{itemize}
\tightlist
\item
  The study of law is the study of predicting what to expect in the
  court of law
\item
  And knowing what the court's expectations are to avoid punishment
\item
  We want to be able to make predictions
\end{itemize}

\subsection{Duties and Obligations}\label{duties-and-obligations}

\begin{itemize}
\tightlist
\item
  Legal Duty is the prediction of what would happen in the court of law.
\end{itemize}

\begin{quote}
a legal duty is nothing but a prediction that if a man does or omits
certain things he will be made to suffer in this way or that way by
judgment of the court; and so of a legal right. {[}3{]}
\end{quote}

\begin{itemize}
\tightlist
\item
  Law is the guise through which we develop our moral judgments.

  \begin{itemize}
  \tightlist
  \item
    Encoded with moral language.
  \item
    Uses rights and duties with malice and intent (moral agency).
  \end{itemize}
\item
  But the power of enforcement is not coextensive with any system of
  morals.
\item
  Rather the study of law is about prediction:
\end{itemize}

\begin{quote}
The prophesies of what courts will do in fact, and nothing more
pretentious, are what I mean by the law.
\end{quote}

\begin{itemize}
\tightlist
\item
  Certain actions are correlated with certain consequences.
\end{itemize}

\subsection{Two Accounts of Malevolent
Motive}\label{two-accounts-of-malevolent-motive}

\begin{itemize}
\tightlist
\item
  Traditional: The defendent being exonorated because he did not possess
  malice (moral sense).
\item
  Modern: The defendent is not exonorated because he caused harm (legal
  sense).
\end{itemize}

\begin{quote}
Signifies the tendency of his conduct under known circumstances was very
plainly to cause the plantiff harm {[}12{]}.
\end{quote}

\section{Lecture Notes -- Oliver Wendell Holmes and Jerome Frank, Legal
Realism}\label{lecture-notes-oliver-wendell-holmes-and-jerome-frank-legal-realism}

• American Legal Realism is a critical position in legal theory inspired
by the work of John Chapman Gray and Oliver Wendell Holmes.

o A bit of background on Holmes: he was a legal scholar and US Supreme
Court Justice. He was also a founding member, with William James,
Charles Sanders Peirce and Chauncy Wright, of the Metaphysical Club,
which was a group that met at Harvard in 1872. It was in this club that
the position that developed into American Pragmatism was first
developed. (Also the subject of a wonderful book by Louis Menand.)

▪ These early pragmatists were metaphysical quietists. ▪ Committed to
the idea that if something doesn't make a difference in practice, then
it's not worth talking about. Differences/distinctions that don't make a
difference in practice are no differences/distinctions at all.

▪ Peirce's pragmatic maxim: • ``Consider what effects, which might
conceivably have practical bearings, we conceive the object of our
conception to have. Then, our conception of those effects is the whole
of our conception of the object.'' (EP1: 132)

▪ James called his pragmatism ``radical empiricism'' • `the only things
that shall be debatable among philosophers shall be things definable in
terms drawn from experience' ▪ We can see these ideas imprinted on
Holmes's thought about ``the law.'' • ``The prophecies of what the
courts will do in fact, and nothing more pretentious, are what I mean by
the law.'' • The law is nothing more than its practical effects, and
those are affected on others by the courts. • Holmes argues for this
position on an empirical basis rather than on conceptual grounds, as
Hart, Austin, and the Natural Lawyers defended their positions.

• He sought to understand both how law shows up in the experience of
those who it affects and how judges actually arrived at their decisions.

o How do those who are governed actually experience the law?

▪ They experience it as an imposition of public force:

• ``You can see very plainly that a bad man has as much reason as a good
one for wishing to avoid an encounter with the public force, and
therefore you can see the practical importance of the distinction
between morality and law. A man who cares nothing for an ethical rule
which is believed and practised by his neighbors is likely nevertheless
to care a good deal to avoid being made to pay money, and will want to
keep out of jail if he can.''

• Holmes, then, maintains the separability thesis, but he does so on
empirical grounds

o He is not a cynic, however\ldots his purpose is to contend that if one
wants to study and practice the law, one must look at it from a
business-like perspective, and this means seeing it as distinct from
law.

• Though law is, he says, ``the witness and external deposit of our
moral lives.''

▪ So, then, we can make our inquiry more precise by asking how the bad
man experiences the law.

• He wants to know what he can get away with, what he can do without
incurring the imposition of public force. • ``But what does it mean to a
bad man? Mainly, and in the first place, a prophecy that if he does
certain things he will be subjected to disagreeable consequences by way
of imprisonment or compulsory payment of money.''

• The bad man is concerned with law as a prediction of likely
consequences, and since those consequences are determined by the court,
he is concerned with law merely as a prediction of how the courts will
decide.

o Holmes examples: contract

• How would Hart respond to this understanding of law as a prediction of
what the courts will decide???

▪ In the making of such predictions (and in the deciding of cases) we do
well to not confuse the moral use of terms and their legal use:

• Example of Malice

o In morality, malice requires ill intent

o In law, it need not, though it may.

▪ ``I think that the law regards the infliction of temporal damage by a
responsible person as actionable, if under the circumstances known to
him the danger of his act is manifest according to common experience, or
according to his own experience if it is more than common, except in
cases where upon special grounds of policy the law refuses to protect
the plaintiff or grants a privilege to the defendant. I think that
commonly malice, intent, and negligence mean only that the danger was
manifest to a greater or less degree, under the circumstances known to
the actor, although in some cases of privilege malice may mean an actual
malevolent motive, and such a motive may take away a permission
knowingly to inflict harm, which otherwise would be granted on this or
that ground of dominant public good.'' • Example of Contracts: o In
morals, contracts or promises are dependent on the internal state of the
persons' minds.

o In law, contracts are purely formal. Whatever the court determines one
to be contractually obligated to on the basis of the executed contract
is what one is obligated to no matter the internal state of one's mind.

o How do (and should) judges decide cases?

▪ We tend to think that judges apply the law, but legal realists think
that judges are mostly in the business of making law.

▪ Why?

• the class of available legal materials is insufficient to logically
entail a unique legal outcome in most cases worth litigating at the
appellate level (the Local Indeterminacy Thesis);

• in such cases, judges make new law in deciding legal disputes through
the exercise of a lawmaking discretion (the Discretion Thesis); and

• judicial decisions in indeterminate cases are influenced by the
judge's political and moral convictions, not by legal considerations.

▪ Examine this passage (p 466)

• Behind the logical form lies a judgment as to the relative worth and
importance of competing legislative grounds, often an inarticulate and
unconscious judgment, it is true, and yet the very root and nerve of the
whole proceeding. You can give any conclusion a logical form. You always
can imply a condition in a contract. But why do you imply it? It is
because of some belief as to the practice of the community or of a
class, or because of some opinion as to policy, or, in short, because of
some attitude of yours upon a matter not capable of exact quantitative
measurement, and therefore not capable of founding exact logical
conclusions. Such matters really are battle grounds where the means do
not exist for determinations that shall be good for all time, and where
the decision can do no more than embody the preference of a given body
in a given time and place. We do not realize how large a part of our law
is open to reconsideration upon a slight change in the habit of the
public mind.

▪ Judges, Holmes thinks, ought to be more explicit about these hidden
drivers of their decisions, and ought to be clear that they make
decisions with social advantage in mind.

▪ This would allow for a re-examination of history and tradition in the
law.

• Turn to p.~470.

• Example of whether we can know that the criminal law does more good
than harm in present circumstances.

\section{Week 6}\label{week-6}

\chapter{Assignments}\label{assignments-2}

\bookmarksetup{startatroot}

\chapter{Summary}\label{summary-1}

In summary, this book has no content whatsoever.

\bookmarksetup{startatroot}

\chapter*{References}\label{references}
\addcontentsline{toc}{chapter}{References}

\markboth{References}{References}

\phantomsection\label{refs}
\begin{CSLReferences}{1}{0}
\bibitem[\citeproctext]{ref-knuth84}
Knuth, Donald E. 1984. {``Literate Programming.''} \emph{Comput. J.} 27
(2): 97--111. \url{https://doi.org/10.1093/comjnl/27.2.97}.

\bibitem[\citeproctext]{ref-smit21}
Smith, Peter. 2021. \emph{An Introduction to Formal Logic}. Second
edition, reprinted with corrections. Monee, IL: Logic Matters.

\end{CSLReferences}



\end{document}
